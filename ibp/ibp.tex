\documentclass{article}

\usepackage{rotating}
\usepackage{graphicx}
\usepackage[utf]{inputenc}
\usepackage{url}

%8.5 pages is 60% 14 pages, which is the max
\title{Industrial Best Practice}


\begin{document}


\author{John Burchell \qquad William Granli \\
		john.a.burchell@gmail.com \qquad william.granli@gmail.com \\
		Computer Science and Engineering  \\
		University of Gothenburg }



\maketitle
\section{Executive Summary, 1p}
Do this last! 

\url{http://unilearning.uow.edu.au/report/4bi1.html}

\section{Problem Formulation, 3p}

As the software industry and the open source movement steadily grow, the number of publicly available APIs is increasing. APIs can improve development speed \cite{stylos2006comparing}, improve software quality \cite{stylos2006comparing} and can increase software reusability \cite{afonso2012evaluating}. There is a consensus amount literature and experts that modifications to deployed APIs can have negative impacts on the users of the API \cite{google_talk} \cite{mcdonnell2013empirical} \cite{robbes2012developers} \cite{henning2007api}. The main reason is that it will require the users to update the code using the API, thus causing a disruption in the application's software ecosystem \cite{messerschmitt2005software}. The most common changes to APIs occur from refactoring \cite{dig2005role} \cite{xing2006refactoring}.

The problem, put simply, is this; how can effective changes be made to deployed APIs with minimal impact upon the ecosystem in which the API is utilised?

%Highlight problem and who it's for here.
Having a stable and well designed API is vital to the success of a company or product, especially if they are built around a particular API. For example, a study on the impact of reliable APIs was undertaken in the Android ecosystem \cite{mcdonnell2013empirical}. The study aimed to discover if APIs with high rates of faults affected the success of an application. The study found that the 50 least popular applications had APIs which were 500\% more error prone, ultimately concluding that there is indeed a correlation between API stability and application success \cite{mcdonnell2013empirical}. 

Due to the negative implications that modifications to APIs cause for users, it is important to understand why changes occur. If the most common pitfalls are known before the development of new APIs begin, these mistakes could be mitigated. Likewise, if common dangers of changes to existing APIs are fully studied and understood, informed changes can be made to lessen the affect on the users. Furthermore, industry and academia can use these lessons learned to create, or enhance existing best practices of API design.

Given that software is prevalent in our lives, it is likely that poor API design changes have affected our lives at some point. One could therefore argue that this problem is important to everyone; customers, businesses and developers.

%This needs fluffing
Our solution is to provide a set of guidelines, or framework with which to follow when making changes to APIs. The framework will prioritise business motives and needs when tackling design decisions.

Exhaustive research has found that there exists very few similar solutions similar to our own. While there are countless books related to the subject [!!!list some?], very few actually aim to address the needs of business and customers; they are primarily focused on the developer.

%Examples of similar products
\subsection{Related Products}
% http://apiary.io/products
Apiary is a popular online API design service. Apiary focuses on a collaborative approach to creating APIs, the tool offers the ability to work with between 5 and 50 contributors on a single project. The unique part of this tool is that it provides a "mock" API by which customers, developers or anyone with access can work with. This stage allows quick prototyping and testing of an API before any real coding has taken place. The tool, while useful, doesn't aim to solve the problem of handling changes to APIs but is instead a good platform to quickly prototype ideas. Apiary also includes some Github integration adding to it's already strong collaboration tools.

%http://www.mashery.com
Mashery provide a more business oriented approach to API development. They offer a wide variety of services with API management being one of them. Quoting their website, they:

Increase productivity through uniformity and accessibility of data
Develop better customer experiences faster, for mobile and web-based applications
Achieve real-time data exchange between platforms and applications
Enhance visibility into data consumption to optimize APIs

While mashery have a more business oriented approach, they do not offer anything resembling a framework for both business and developers to use when making API design decisions.

%http://swagger.io/
Swagger 2.0 is another online design tool for APIs. Swagger has the added bonus of including a "Generator" for creating some basic library code for users to build upon. Swagger, online Apiary, is primarily used for RESTful APIs. Similarly to the previously mentioned companies, Swagger is more of a tool solution rather than a guideline or framework.






\begin{itemize}
	\item Description of what problem you are trying to solve
	\item The description should highlight why the problem is important, for whom it is important and how it advances the state of the art/company business
	\item Description of similar solutions available for the market
\end{itemize}





\section{Plan and Roadmap, 5p}
As mentioned in Section 2, the effects modifications to APIs have on the platform’s ecosystem are explored as being negative. However, there are no existing systematic guidelines or frameworks that take the ecosystem into account for how to best modify an API. Our proposed process for creating such a framework can be seen in Figure \ref{fig:roadmap}. This is an initial version of the roadmap \cite{!!!roadmap}, and it is advised that it is updated during the process. The main reason for this is that it is important to adapt to changes in the market, and if new needs are identified, the strategy should be redesigned to meet them. 

\subsection{Designing the Roadmap}
The roadmap was developed using the standard T-Plan process \cite{!!!roadmap}. As such, identifying the market need was done as a part of the initial step. Following that, the products and technology needed to fill that gap were identified. As a final step the dependencies and ordering of the identified activities were sorted out which resulted in the current version of the roadmap. An overview of the process used to create the roadmap can be seen in Table \ref{tab:proc}

\begin{table}[ht]
\centering
\begin{tabular}[ht]{|c|l|l|}
\hline
\textbf{\#} & \textbf{Phase} & \textbf{Description} \\
\hline
1 & External & Identify the market needs \\
\hline
2 & Products & Identify required studies \& frameworks \\
\hline
3 & Technology & Identify required knowledge \\
\hline
4 & Organise & Identify what pull factors require which push factors \\
\hline
\end{tabular}
\caption{Roadmap Design Process}
\label{tab:proc}
\end{table}

\subsection{Knowledge Development}
Since we have made the assessment that the market need will remain relatively stable, careful research is essential to be able to deliver a product that meets the market needs. The main way of performing the initial study, to be able to increase the knowledge, will be through studying the existing literature. The reason for this is that three fields (API design, software evolution and software ecosystems) are currently well-explored, but the correlation and relationship between them has not yet been identified. Therefore, these fields will be explored in literature and a set of ``best practices" will be gathered from each field. 

In addition to the literature review, a case study will be conducted to gather information from a state of the art company. The goal of the data collection made in the case study, will also be to gather a set of best practices for each of the three fields. This will be achieved by analysing the changes between two versions of the company's platform API. 

\subsection{Implementing the Framework}
To be able to create a framework which uses as much as possible of the knowledge and technology that exists in the world, the results from the literature review and the case study will both be analysed and compared. A set of best practices based on both data collections will be defined and become the basis for the framework. That we bring in important factors from each of the three fields in the discussion is of the highest importance. The goal of this phase is to create an Minimum Viable Product (MVP) that can be adopted and evaluated at the case company. 

\subsection{Adoption and Evaluation}
The final stage of the roadmap will take a more iterative approach, where the MVP will be continuously evaluated and improved at the case company. The main way of evaluating the success of the framework will be to measure if there is an decrease in the negative effects that the API has on its ecosystem, while it still is able to remain its original complexity and usability. 

\begin{sidewaysfigure}[p]
\centering
\includegraphics[width=220mm]{RoadMap.png}
\caption{The full lines denote in which way the organisation should ``logically'' view the roadmap. The dotted lines denote that one activity will lead to another. }
\label{fig:roadmap}
\end{sidewaysfigure}
%TODO: Add colours+graphix so that it's innovative and stuff

\section{Dissemination Plan, 2p}
This section will describe the dissemination plan for our framework. 

\subsection{End-users}

Two major end-users of our framework have been identified, 1) open-source projects and 2) web development APIs. The reason open-source projects have been identified as potential users of our API is that they generally have a good-will mindset when it comes to developing the software engineering industry. Their goals are not set on making profit, but delivering a product that is as good as possible. The open-source community also takes care of itself and thus it cares about the well-being of its ecosystems. The second area where this could have an especially great impact would be in the area of web development APIs, since they are widely used by a great amount of users. Another reason that web APIs would be impacted is that very clear boundaries between the ecosystems exist. The toolchain in web development is usually referred to as a ``stack" which literally is a set of frameworks and APIs that you use together to develop your application. 

\subsection{Presentation and Communication}
The main way of conveying our message would be through the research community. The initial studies that will be conducted (as mentioned in Section 3) will lead to a number of published research papers. These will be used to create awareness and attention of our framework. The papers may also lead to other researchers building on our research which would only help our cause which is to improve APIs everywhere. 

In addition to reaching out to the research community, conferences related to web development would be targeted especially. One example of such a conference is JDays, where many APIs that are developed in JavaScript are discussed. The main reason for targeting such conferences is that a large part of the important persons of the community attend. This would help to create further awereness and publicity. 

The last way of presenting our framework would be through directly targeting a set of companies that operate in the same software ecosystem. One example of that would be to target a company which develops a game engine, and at the same time contact the most prominent game developers which utilise this game engine. If the company developing the game engine realises that there is large interest from the main users of their API, it is much more likely that they would be interested in using our API. 


\section{Impact Assessment, 2p}
This section will describe the impact that our framework will have. 

\subsection{Background}
The key impact of our new API framework would be that it would transform the way companies and organisations motivate their design decisions for APIs. Today's software engineering companies mainly take into account their own agendas, and APIs are commonly changes for refactoring reasons or to reduce complexity. The most important factor when designing or changing APIs is, however, the users of the API and their needs. In the recent years, this has been realised by the software engineering industry. This has lead to less deprecations in the code of the API users. 

\subsection{Impact on Technology and State of the Art}
The next step that we propose the industry takes, is to weigh the needs of the whole ecosystem in the assessment when creating APIs. This would lead to a more homogeneous environment where API platforms and API users can change their code jointly, to reduce the risk of deprecating for example methods. The impact of this could even be that companies start changing and improving their APIs to help the ecosystem, and thus causing a domino effect in the ecosystem which can even spread to other ecosystems. 

\subsection{Impact on Society}
If companies start developing their APIs to fit the needs of ecosystems, it would not only have a great impact on these specific isolated environments; it would develop the whole software engineering industry. In modern software development, very few applications are written without the use of APIs, and this means if APIs become better; it can increase the performance of software development. It might be hard to envision that companies would actually spend resources to help their ecosystem without helping themselves (directly). But examples of where similar things are happening can already be found today, and one example is that Ericsson are spending a lot of resources to develop open-source software such as Eclipse and Papyrus. If this would catch on, it would basically lead to a world where all software has better quality and is developed faster. 

\section{Summary, 1p}

\begin{itemize}
	\item Account of how you can use the knowledge from the course in your further work
\end{itemize}




\end{document}