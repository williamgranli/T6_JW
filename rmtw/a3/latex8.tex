
%
%  $Description: Author guidelines and sample document in LaTeX 2.09$ 
%
%  $Author: John Burchell & William Granli $
%  $Date: 2015/01/21 15:20:59 $
%  $Revision: 1.0
%

\documentclass[times, 10pt,twocolumn]{IEEEtran} 
\usepackage{latex8}
\usepackage{times}

%\documentstyle[times,art10,twocolumn,latex8]{article}

%------------------------------------------------------------------------- 
% take the % away on next line to produce the final camera-ready version 
\pagestyle{empty}

%------------------------------------------------------------------------- 
\begin{document}

\title{Review of Group 8 by Group 1}

% author names and affiliations
\author{\IEEEauthorblockN{
  John Burchell\IEEEauthorrefmark{1} and
  William Granli\IEEEauthorrefmark{1}}

  \IEEEauthorblockA{\IEEEauthorrefmark{1}john.a.burchell, william.granli@gmail.com}

}


\maketitle
\thispagestyle{empty}





%------------------------------------------------------------------------- 

%Deficiencies introduction model

​\Section{Introduction}
*5 references must be used to support claims in Introduction
Use parts of Abstract and Intro from CM. 

Coping with change in the software development domain is difficult at best, this is further exasperated by trying to apply changes that do not mesh well with existing development paradigms. How then, do large companies, especially those utilising multiple development paradigms, cope with this kind of change?

\SubSection{Problem Statement}
The purpose of this paper is therefore to determine whether an organisation's development paradigm should be the basis for deciding which process and design changes to adopt. 

%Maybe not this bit-----------------------------------------------------------
To explore this area, we focus on process and design practices that have been introduced at Ericsson EPG.
%Maybe not this bit-----------------------------------------------------------


\SubSection{Existing Literature}
Use parts of the CM report. 
Most relevant is probably Jani's but could find stuff in the other's too. 

Little existing literature.
There are gaps: see: nothing about applying changes to two different paradigms at the same time.

\SubSection{Deficiencies in Literature}
What is missing

\SubSection{Motivation for Research}
Academia: Has not been done before. (I.e. all CM papers talk about implementing something in a homogenous environment). 
Industry: It's useful to Ericsson. They have had problems implementing changes and keeping up with the competition. It might be good to look up other companies which use MDSD and make a case for them as well. 

\Section{Purpose of the Study}
It's an exploratory study. We want to find out how CM is done in such environemnts, if it's successful and how it can be improved. 

\SubSection{Research Question(s)}
RQ: How is change managed in a company which has a split development strategy (e.g. mdsd+manual)?
SRQ1: How can it be done better blabla?
SRQ2: How can a company determine if a change will conflict with a current paradigm?

\Section{Case Company Description}
TBA

\Section{Methodology}
We aim to conduct an exploratory study. ("finding out what is happening, seeking new insights and generating ideas
and hypotheses for new research.", runesson host)

Data Triangulation:
Data (source) triangulation—using more than one data source or collecting the same
data at different occasions.
Observer triangulation—using more than one observer in the study.
Methodological triangulation—combining different types of data collection methods,
e.g. qualitative and quantitative methods.
Theory triangulation—using alternative theories or viewpoints.

The design will be flexible based on the data collected (runesson host). 
	Come up with examples on how x data can affect the study. 


Why case study? 
Case studies offer an approach which does not need a strict boundary between the studied object and its environment; perhaps the key to understanding is in the interaction between the two? (runesson host)

Steps \& Phases
1. Case study design: Objectives are defined and the case study is planned. 
Runesson+Höst mention the importance of carefully planning case studies, despite them being of the flexible type. 
Planning consists of which methods to use for the data collection, which departments to visit, which documents to read, who to interview, when interviews should be conducted, etc. These can be formulated in a case study protocol. 
It should contain: 
Objective: What to achieve?
The case: What is studied?
Theory: Frame of reference?
Research questions: What to know?
Methods: How to collect data?
Selection strategy: Where to seek data?

2. Preparation for data collection: procedures and protocols for data collection are defined
3. Collecting evidence: Execution with data collection on the studied case
A direct method (first degree) (Lethbridge et al. 2005) will be used. (Interviews).
Third degree (Lethbridge et al. 2005) can also be used. I.e. analyse work artifacts. (The time-to-market based on requirments thing). (More info page 19)

Mix of open/closed questions + semi-structured. 




4. Analysis of collected data
Analysis will be done in tandem with the data gatheritng to be able to iteratively improve the data collection process+tools

Hypothesis generation will be an important part of the data analysis, since the outcome/goal of the study is not clear. 


5. Reporting
(Descriptions on page 8 of the pdf in runessson host paper)


Case study protocol
\#	Goals and Scope		Data collection and filtering	Analysis and presentation	Interpretation and Improvement
(More info on Page 11 of the host pdf)

Mention stuff about ethical problems (page 12)


\begin{itemize}
\item Describe and explain the research strategy (i.e. case study)
\item Describe and explain the investigation methods that will be used
\item Motivate the choice of case study using relevant references
\item Describe the data collection procedures, sampling method(s), number of subjects etc.
\item Describe the subjects and how they were selected
\item Create questions for your interviews (include the questions in your report)
\item Describe and motivate the method for analyzing the data
\item Discuss validity threats
\end{itemize}

\Section{References}
Make real ref list later


\Section{Acknowledgements}
Write who did what





\end{document}

