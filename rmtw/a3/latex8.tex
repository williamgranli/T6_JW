
%
%  $Description: Author guidelines and sample document in LaTeX 2.09$ 
%
%  $Author: John Burchell & William Granli $
%  $Date: 2015/01/21 15:20:59 $
%  $Revision: 1.0
%

\documentclass[times, 10pt,twocolumn]{IEEEtran} 
\usepackage{latex8}
\usepackage{times}

%\documentstyle[times,art10,twocolumn,latex8]{article}

%------------------------------------------------------------------------- 
% take the % away on next line to produce the final camera-ready version 
\pagestyle{empty}

%------------------------------------------------------------------------- 
\begin{document}

\title{Review of Group 8 by Group 1}

% author names and affiliations
\author{\IEEEauthorblockN{
  John Burchell\IEEEauthorrefmark{1} and
  William Granli\IEEEauthorrefmark{1}}

  \IEEEauthorblockA{\IEEEauthorrefmark{1}john.a.burchell, william.granli@gmail.com}

}


\maketitle
\thispagestyle{empty}

%------------------------------------------------------------------------- 

%Deficiencies introduction model

​\Section{Introduction}

Coping with change in the software development domain is difficult at best, it is further exasperated by trying to apply changes that do not mesh well with existing development paradigms. How then, do large companies, especially those utilising multiple development paradigms, cope with this kind of change?

\SubSection{Problem Statement}
The purpose of this paper is therefore to determine whether an organisation's development paradigm should be the basis for deciding which process and design changes to adopt. 

%Maybe not this bit-----------------------------------------------------------
To explore this area, we focus on process and design practices that have been introduced at Ericsson EPG.
%Maybe not this bit-----------------------------------------------------------


\SubSection{Existing Literature}
According to literature, changes in companies are commonly implemented following either Software Process Improvement (SPI)\cite{pettersson2008practitioner}\cite{unterkalmsteiner2012evaluation} or change management (CM) strategies. SPI is an attempt to alter an existing company or organisations software process, the main intention of which is to increase product quality, time to market all the while reducing inefficiency and other obstacles for creating good software. 

Similar to SPI is the concept of Evidence Based Software Engineering \cite{dyba2005evidence}. EBSE can be used as support for SPI, as suggested by the creators of the strategy. The main aim of which is to find a technology which is suitable for use and to then implement it in an organisation.

CM on the other hand could be seen as an umbrella for many theories and strategies. Such an example of this is the concept of innovation values fit by [CITE].

In 2002 Hall, Rainer and Baddoo \cite{hall2002implementing} published a report wherein they illustrated results of a study which looked at 85 companies throughout the UK. The study focused mostly on how the companies perform SPI and how effective their efforts were. The authors discovered that, while most companies do attempt SPI, many are unsuccessful due to a combined lack of evaluation and funding.

A further, more specific example is a study conducted by Staron \cite{staron2006adopting}. In this study a case study is conducted with two companies wishing or in the process of adopting MDSD. In the findings, Staron provides a set of conditions for adopting MDSD, one being the compatibility of MDSD and existing principles in the company. Staron states that the adoption of this new process must be usable by the previous processes in place \cite{staron2006adopting}. He concludes by stating that this compatibility is especially important for larger companies due to the heavy effort required to implement the change. In a second study \cite{staron2008transitioning}, Staron provides possible reasons for the adoption of MDSD to fail. He suggests that the large investment in time and money, immature technology and organisational resistance are all factors to overcome in the adoption of new processes.

However, there are is no existing research that aims to tackle the problem of applying the same change to an organisation that utilises two differing development paradigms.

\SubSection{Motivation for Research}
Performing this study would therefore be an attempt at filling in a gap in literature. Academia would be interested in this for its own merit or, most likely, as a stepping stone towards further research.

Industry is just as likely to be interested as change is something they face regularly. If this study can help to shed some light about specific methods of to help accommodate change. The study will be conducted in tandem with an organisation, to which the results would also be of great value. The organisation in question has had problems in the past with implementing effective change to keep up with their competition. 

Furthermore, companies that operate in a similar, or even a different manner would most likely be able to appreciate the efforts made in the study. In the most hopeful of cases, applying the lessons learned to their own situations.

Given that MDSD has seen a considerable rise in adoption by many companies\cite{millermda}, the results of the research would undoubtedly be of use to these companies also.

\Section{Purpose of the Study}
It's an exploratory study. We want to find out how CM is done in such environemnts, if it's successful and how it can be improved. 

\SubSection{Research Question(s)}
RQ: How is change managed in a company which has a split development strategy (e.g. mdsd+manual)?
SRQ1: How can it be done better blabla?
SRQ2: How can a company determine if a change will conflict with a current paradigm?

\Section{Case Company Description}
TBA
\newpage

\Section{Methodology}


\SubSection{Case Study Design}
The study has been conducted with an exploratory goal in mind. Analysing how Company X tackle the problems related to the research questions has been the most central part of the study. Because of the exploratory nature of the study, a flexible process design has been used. The study has been carried out using an iterative approach, where the data collected, and the result gathered from analyzing the data has been the main driver for shaping and improving the scope and goal of the study. The research has been conducted using a case study approach. The reason for that is that the goal of our study is to analyze MDSD, which is a common practice in software engineering, in a very specific setting. This line of thought is supported by Runesson \& Höst in CITE XXX. Runesson \& Höst also outline that a case study approach is well-suited for exploratory studies CITE XXX. 

In the initial stage of the study, careful planning was conducted. The two major goals of this phase were to a) find meaningful objectives and design for the research, and b) facilitate and design the interviews. Although a flexible approach was taken, major effort was spent on planning the study. Most of the planning related to communication with Company X, to facilitate the interviews. A timeline for the interviews was created and established with Company X. The objective of analysing change implementation in enviroments which are split into by two different development activities was also defined during this stage. 


\SubSection{Preparation for Data Collection}
During this stage the interviews were designed. The initial design of the interview was pilot tested on three employees of Company X. The results from the pilot tests were used to test if the interviews would fulfil the objective of the study. The number of subjects for the main interviews were 15. They were sampled randomly from employment records in Company X. 

During this stage, the all ethical precautions of the study were taken. It was established that Company X wanted to remain anonymous and that all interviews must happen on voluntary basis. An agreement on to which extent Company X's documentation could be used was also made during this stage. 

\SubSection{Data Gathering}
The data has mainly been gathered through first degree interviews CITE LETHBRIDGE 2005 XXX. The subjects of the interviews can be divided into three sub-groups: a) Managers b) Developers c) Change Agents. Data triangulation has been used to compare and verify the results based on which role the interviewee had. The reason this was done was to reduce risk of, for example, change agents having a positive bias towards previously implemented changes. The interviews have been carried out using a mix of open-ended and close-ended questions, and a semi-structured approach has been used. The fact that the interviewees have widely differing backgrounds and roles was the main reason for not using a structured approach. A semi-structured approach was chosen over a unstructured approach to be able to get results that are closely related to eachother, so that meaningful comparison and analysis can be carried out. Company X's change process documentation has also been analysed to provide further depth to the analysis. Please see Appendix A for a full list of interview questions.

\SubSection{Data Analysis}
The data analysis of the interviews has been done in tandem with the data gathering to be able to iteratively improve the data collection process. This data analysis has had the main purpose of generating hypotheses and to refine the research questions. The interviews were analysed by manually reading and comparing answers. Spreadsheets and tables were also used to provide an overview of the results. 

When all the interview data had been collected, the data analysis goal was switched to focus on confirmation of the hypothesis. As mentioned in Section IV. C, data triangulation was the main tool used to confirm the hypothesis. 


\Section{Threats to Validity}







\begin{itemize}
\item Create questions for your interviews (include the questions in your report)
\item Discuss validity threats
\end{itemize}

\Section{References}
Make real ref list later


\Section{Acknowledgements}
Write who did what





\end{document}

