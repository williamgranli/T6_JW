
%
%  $Description: Author guidelines and sample document in LaTeX 2.09$ 
%
%  $Author: John Burchell & William Granli $
%  $Date: 2015/01/21 15:20:59 $
%  $Revision: 1.0
%

\documentclass[times, 10pt,twocolumn]{article} 
\usepackage{latex8}
\usepackage{times}

%\documentstyle[times,art10,twocolumn,latex8]{article}

%------------------------------------------------------------------------- 
% take the % away on next line to produce the final camera-ready version 
\pagestyle{empty}

%------------------------------------------------------------------------- 
\begin{document}

\title{Review of Group 9 by Group 1}

% author names and affiliations
\author{John Burchell and William Granli \\
john.a.burchell@gmail.com, william.granli@gmail.com}



\maketitle
\thispagestyle{empty}


%------------------------------------------------------------------------- 


\Section{Introduction}

Most of the way into the introduction and I don't know what the paper is really about, other than it's related to MDD. Maybe you should consider splitting the introduction upon into smaller sections, such as having a general introduction to the problem, a problem statement and then some related work to show why this is a problem?

Good paragraph with Orlikowski \& Gash, well referenced and backed up with another good paper.

\Section{Purpose of the Study}
As mentioned before, this should probably be mentioned a little earlier than here. Some of the wording could be improved here, it's a little repetitive. 

Again, the structure is something that I would consider changing here, the research questions should be separated from the text, they should be very easy to locate and to find, maybe put them in a list? Also, your question doesn't even have a question mark at the end.

I would suggest moving most of the text out of this section and into others. For example, I would move the supporting comments and reasons before you state the actual research questions. This applies to the sub questions also.


\Section{Methodology}

The first time that you mention the research strategy is in the methodology itself, while I know that the assignment was an SLR, it's probably a good idea to mention it before the methodology in passing, for example in the problem statement.

Good arguments and explanation of the SLR, however, it might flow better to start with explaining the SLR, explaining why it was picked and then the validity threats, on that note validity threats should be their own section.

\Section {Database and search keywords}

It might be a little more prudent to have a nice list of the words used and to put it into the appendix rather than having it in the middle of the papers text. Also for the amount of papers found, it would look and read much more easily if they were tabulated, i.e. you have each database used with the amount of papers found.

Little inconsistency with the search term, was it ``Benefits'' or benefits?

\Section {Inclusion Exclusion Criteria}

The criteria were good but it's a bit difficult to read when it's all put into a single block. This section could do with some more substance, for example, are you including only positive papers? Are you excluding negative papers? If including the positive, how are you going to alleviate the research bias that this brings with it?

\Section {Data Extraction}

The data extraction section is ok overall, however the mentioning of the pen and paper tools feels a little strange. Also, you mention drawbacks and benefits of these tools and that they should be taken into consideration. What are the drawbacks and benefits of the tools you picked?

Furthermore, is each reviewer going to review each paper twice, or once? You should probably include this and why for either reason decided. The section detailing the forms is good, if you have time it might be nice to see an example of the form that you might use.

\Section {Data Synthesis}
I'm not sure what you intended this section to show or do, but all I understood is that you stored the data. What did you actually do to synthesise the data? How was it combined? It needs much more information.

\Section {Quality Assessment}
Good points about the criteria and quality, but what about other kinds of threats? Other validity threats?

\end{document}

