
%
%  $Description: Author guidelines and sample document in LaTeX 2.09$ 
%
%  $Author: John Burchell & William Granli $
%  $Date: 2015/01/21 15:20:59 $
%  $Revision: 1.0
%

\documentclass[10pt,twocolumn]{article} 
\usepackage{latex8}
\usepackage{url}

%\documentstyle[times,art10,twocolumn,latex8]{article}

%------------------------------------------------------------------------- 
% take the % away on next line to produce the final camera-ready version 
\pagestyle{empty}

%------------------------------------------------------------------------- 
\begin{document}



\title{Assignment 5}

% author names and affiliations
\author{John Burchell and William Granli \\
john.a.burchell, william.granli@gmail.com}


\maketitle
\thispagestyle{empty}

%------------------------------------------------------------------------- 

%Deficiencies introduction model


​\Section{Introduction}

\SubSection{Statement of the Problem}
Companies don't know if adopting MDD will be beneficial in terms of money and improved software quality.

\SubSection{Existing Literature}

\SubSection{Motivation for Study}

*5 references only needed for above (re-use from a1)

\Section{Purpose of the Study}
Purpose leading into RQs

\SubSection{Research Questions}
\begin{itemize}
\item \textbf{RQ1:} In which situations should companies expect that changing to MDSD would lead to positive effects? 
\item \textbf{RQ1.1:} In which domain(s) is MDSD the most successful? 
\item \textbf{RQ1.2:} Which software development process(es) has been most successfully used in conjunction with MDSD?
\item \textbf{RQ1.3:} What are the characteristics of the systems of the companies that successfully use MDSD?
\end{itemize}

\begin{itemize}
\item They are probably not worded optimally... Here're some alturnutuves:
\item In which situations should companies expect MDD to be successfully adopted?
\item In which situations should companies expect that the adoption of MDD would lead to a performance boost?
\end{itemize}


\Section{Methodology}

\begin{itemize}
\item Describe and explain the research strategy (i.e. SLR) and investigation methods that will be used for the study. 
\item Motivate your choice of experiment using relevant references. 
\item Search string including keywords and used databases
\item inclusion/exclusion criteria
\item data extraction
\item data synthesis
\item quality assessment
\end{itemize}

kithcneham start

\textbf{Planning the review}
\begin{itemize}
\item Identification of the need for a review (5.1)
Check other SLRs in the field and evaluate them
\item Commissioning a review (5.2)
not needed
\item Specifying the research question(s) (5.3)
this is very important :)
the search process must identify primary studies that address the research questions
the data extraction process must extract the data items needed to answer the questions
the data analysis process must synthesise the data in such a way that the questions can be answered.

read the questions in 5.3.1 to come up with similar ones for readability

\item Developing a review protocol (5.4)

\item Evaluating the review protocol (5.5)
\end{itemize}

\textbf{Conducting the review}
\begin{itemize}
\item Identification of research (6.1)
\item Selection of primary studies (6.2)
\item Study quality assessment (6.3)
\item Data extraction and monitoring (6.4)
\item Data synthesis
\end{itemize}

\textbf{Reporting the review}
\begin{itemize}
\item Specifying dissemination mechanisms (7.1)
\item Formatting the main report (7.2)
\item Evaluating the report (7.3)
\end{itemize} 

\bibliographystyle{latex8}
\bibliography{latex8}


\Section{Acknowledgements}
U w0T m8


\end{document}

