
%
%  $Description: Author guidelines and sample document in LaTeX 2.09$ 
%
%  $Author: John Burchell & William Granli $
%  $Date: 2015/01/21 15:20:59 $
%  $Revision: 1.0
%

\documentclass[10pt,twocolumn]{article} 
\usepackage{latex8}
\usepackage{url}
\usepackage{tabularx}
\usepackage{appendix}
\usepackage{pbox}


%\documentstyle[times,art10,twocolumn,latex8]{article}

%------------------------------------------------------------------------- 
% take the % away on next line to produce the final camera-ready version 
\pagestyle{empty}

%------------------------------------------------------------------------- 
\begin{document}




\title{Wu-Tang Clan}

% author names and affiliations
\author{John Burchell and William Granli \\
john.a.burchell@gmail.com, william.granli@gmail.com}


\maketitle
\thispagestyle{empty}

%------------------------------------------------------------------------- 



​\Section{Introduction}
Ever since the first computer programs were written, researchers and developers have been working on new programming concepts in order to decrease the level of complexity needed to develop programs. The abstraction level has been increased with help of compilers, object-oriented programming and Unified Modelling Language (UML). The current trend within software engineering (SE) is Model-driven Software Development (MDSD) which uses UML to generate object-oriented code \cite{staron2006adopting}. 

The benefits of MDSD have consistently been reported in academia \cite{staron2006adopting} \cite{volter2013model}. Many large companies such as Ericsson and ABB \cite{staron2006adopting} also use MDSD as a part of their development chain. In industry, however, MDSD has struggled to make a complete breakthrough and it is mostly used for niche systems such as embedded and safety-critical systems.

\SubSection{Statement of the Problem}

The selection of development process, programming language and programming paradigm has a high impact on the development speed depending on what the the system requirements are and in which environment the system is developed. Selecting for example MDSD for the wrong system and setting may have a very negative impact on development speed, quality and maintainability. 

\SubSection{Existing Literature}

Volter et al. \cite{volter2013model} state that successful adoption of MDSD should increase development speed, increase software quality, improve reusability and improve manageability through abstraction.

Whittle and Hutchinson \cite{whittle2012mismatches} conducted a study whereby they examined 17 companies who were adopting MDSD. While they agreed that the successful adoption of MDSD can improve productivity through code generation, there are many pitfalls which can cause the adoption to fail. Whittle and Hutchinson found that in companies where the change was implemented bottom up, there was a higher degree of success. On the contrary to this, when change was top down, as suggested by academia and education, the adoption was often not as successful. A focus on Domain Specific Languages (DSLs) is what Whittle and Hutchinson suggest helps MDSD become successful in a company setting.

A further, more specific example is a study conducted by Staron \cite{staron2006adopting}. In this study a case study is conducted with two companies wishing or in the process of adopting MDSD. In the findings, Staron provides a set of conditions for adopting MDSD, one being the compatibility of MDSD and existing principles in the company. Staron states that the adoption of this new process must be usable by the previous processes in place \cite{staron2006adopting}. He concludes by stating that this compatibility is especially important for larger companies due to the heavy effort required to implement the change. In a second study \cite{staron2008transitioning}, Staron provides possible reasons for the adoption of MDSD to fail. He suggests that the large investment in time and money, immature technology and organisational resistance are all factors to overcome in the adoption of new processes.

Mattsson et al. \cite{mattsson2009linking} discuss the correlation between MDSD and software architecture. The idea that MDSD is beneficial if performed correctly is presented, and one important factor discussed is the need of automatic enforcement of architectural rules \cite{mattsson2009linking}. If this has to be performed manually, it will become a bottleneck in large MDSD projects \cite{mattsson2009linking}.

\SubSection{Motivation for Study}

Academia appears to be united in the opinion that MDSD is an improvement compared with traditional software development \cite{staron2006adopting} \cite{volter2013model}. However, why do many large software companies choose to not adopt MDSD?

This gap in the literature is the motivation for our study. There is therefore a need to methodically investigate the existing literature in which studies of the adoption of MDSD, advantages and disadvantages of MDSD are explored. Not only this but how and when should MDSD be adopted and how it can be ensured to succeed.

\SubSection{Research Questions}
\begin{itemize}
\item \textbf{RQ1:} In which situations should companies expect that changing to MDSD would lead to positive effects? 
\item \textbf{RQ1.1:} In which domain(s) is MDSD the most successful? 
\item \textbf{RQ1.2:} Which software development process(es) has been most successfully used in conjunction with MDSD?
\item \textbf{RQ1.3:} What are the characteristics of the systems of the companies that successfully use MDSD?
\item \textbf{RQ1.4:} Which change strategy has most been successful in implementing MDSD? 
\end{itemize}

\Section{Methodology}
The method used to attempt to answer our research questions was a Systematic Literature Review (SLR) \cite{kitchenham2007guidelines}. An SLR was the most appropriate research methodology for two main reasons. Firstly, we wished to summarise the existing evidence of when MDSD is applicable in industry and secondly to fill gaps in literature for examples of when MDSD is not applicable \cite{kitchenham2007guidelines}. For these same reasons, other methodology techniques were not as applicable. For example, a quantitative study was not appropriate as we were not trying to prove, disprove or generalise a hypothesis. A qualitative study would have been useful if we were interested in exploring a hypothesis. For similar reasons a case study would also not have been applicable as we were not concerning ourselves with how MDSD works in a real setting. Likewise, as industry and academia seem to differ in opinion, there is an implication that there could have been be bias in both sides of the argument. By performing an SLR we helped to alleviate some of this bias but we did not eliminate it entirely \cite{kitchenham2007guidelines}.

It is important to note that did not perform either a Systematic Mapping Study nor a Tertiary Review of (XXX cite?) existing literature. There was sufficient evidence in existence for the applicability of MDSD such that we could rule out the need to perform a Systematic Mapping Study. Furthermore, there were insufficient SLRs that had been conducted prior to our study, thus, a Tertiary Review \cite{kitchenham2007guidelines} was not possible nor appropriate at that time.


\SubSection{Planning the Study}
Initially, existing SLRs were identified and evaluated, with the goal of mapping and identifying the gap in the existing literature. All identified SLRs \cite{santiago2012model} \cite{dias2007survey} \cite{loniewski2010systematic} \cite{nicolas2009generation} were, however, researching the usage of model-driven techniques in other fields, such as traceability management \cite{santiago2012model}, testing \cite{dias2007survey} and requirements engineering \cite{loniewski2010systematic} \cite{nicolas2009generation}. Based on these results, the need to evaluate previous research on model driven \textit{software} development was identified. 

The next step was to define the research questions. The main research question was constructed to create a wide scope to be able to investigate which characteristics are most vital to consider when adopting MDSD. The sub-questions were constructed to provide more detailed answers for specific questions that are valuable to the industry and companies who are considering to implement MDSD. 

\subsubsection{Review Protocol}
Subsequently, the review protocol was defined. The list of databases included can be seen in Table I. The included databases have been chosen based on recommendations in \cite{brereton2007lessons}. SpringerLink was included as advised in \cite{kitchenham2007guidelines}. The databases were divided evenly between the two researchers. 

\begin{table}[ht]
	\centering
	\begin{tabular}{|l|l|} 
		\hline
		\textbf{Database} & \textbf{Link}  \\
		\hline
		IEEExplore &  Granli \\
		\hline
		ACM Digital library &  Granli \\
		\hline
		Google Scholar &  Granli \\
		\hline
		Citeseer Library & Granli \\
		\hline
		Inspec &  Burchell \\
		\hline
		ScienceDirect & Burchell \\
		\hline
		EI Compendex &  Burchell \\
		\hline
		SpringerLink & Burchell \\
		\hline
	\end{tabular}
	\caption{Databases included in review}
\end{table}

The search terms used can be found in Appendix A. The main search terms are related to MDSD, but related terms such as Model Driven Engineering and code generation were also included. 

The inclusion criteria were defined as follows: \newline
\begin{itemize} 
\item Studies including examples about the application of MDSD in industry
\item Studies carried out using a qualitative methodology
\end{itemize}

The exclusion criteria were defined as follows: \newline
\begin{itemize}
\item Informal papers (no defined research questions, no methodology)
\item Non-peer reviewed papers (books, white papers etc.)
\item Studies carried out using a quantitative methodology
\end{itemize} 

The protocol was evaluated by Ioannis Kellaris and Patrik B\"ackstr\"om. The main objective of the review was to ascertain that the current protocol would properly address the research questions. Each step of the review protocol was evaluated based on this criteria. 

\SubSection{Conducting the Study}

To ensure that our search criteria had be meaningfully defined, a small trial study was performed to guarantee that our criteria could find six predetermined and well known papers which are relevant to our study. As Kitchenham \cite{kitchenham2007guidelines} suggests, papers from other disciplines will be included in the literature review. 

Possible publication bias was taken into account during the study. From our trial study and previous research into related work, it became clear that there existed bias towards the adoption of MDSD; there were few papers which spoke ill of it. To help alleviate this we checked, but did not include, grey literature (non-formally published papers) and conference proceedings. We also contacted researchers and change agents with experience of implementing MDSD to discover if they had unpublished results which could balance the positive bias. 

\subsubsection{Study Selection}

In order to correctly collect all the relevant papers, a particular search process had to be defined. The table below shows the correlation between the types of sources and how they were searched.

\begin{table}[ht]
	\centering
    \begin{tabular}{ | l | p{3.5cm} |}
    \hline
    \textbf{Data Source} & \textbf{Data Collected} \\ \hline
    Digital Sources & Database Name \newline Date of Search \newline Year span \\ \hline
    Conference Proceedings & Conference Title \newline Conference Name \newline Journal Name \\ \hline
    Unpublished Studies & Names of those contacted \newline Research websites\newline searched (Date and URL) \\
    \hline
    \end{tabular}
	\caption{Searching Strategies}
\end{table}

The digital libraries, mentioned in Table 1, were chosen for their extensive catalogues, reliability and availability of the data. Hand searching of journals was not performed as we decided to limit the scope of the search to digital libraries, conferences and other sources.

Given that MDSD is a relatively new process, we decided to limit the search starting to papers published 1990 to the current date. We feel that this limitation on the search did not hinder the results as we were unable to find any papers published prior to this date concerning MDSD.

The selection criteria for the papers were refined during this stage. After applying the initial exclusion and inclusion criteria, exclusions and inclusions were based on practicality. Below is a short list of these criteria.

\begin{itemize}
\item Language
\item Date of publication
\end{itemize}

\subsubsection{Quality Assessment}

The quality of the papers was also considered before they were examined for use in the study. Determination of paper quality occurred after initial exclusions and inclusions had been performed. These additional quality checks sought to eliminate papers with excessive bias, low quality and the unreliable results.

%Maybe this needs more backup, or some more oomph?

To assess the reliability of the inclusions and exclusions, Fleiss' Kappa index was applied on both researchers findings. All conflicts regarding to acceptance of papers was discussed and resolved prior to their inclusion in the SLR. The initial Kappa rating achieved was 0.78, which lies in the band 0.61 - 0.80 indicating a strong alignment between the researchers. 

Furthermore, as per the suggestion from Kitchenham \cite{kitchenham2007guidelines} after the application of the inclusion and exclusion criteria any paper failing to meet these criteria was retained and will be shown in the results of the SLR.

%Skipping some steps of this as they don't feel that relevant, discuss with willy.

\subsubsection{Data Extraction}

In order to accurately and objectively collect all data, the use of data collection forms is highly recommended \cite{kitchenham2007guidelines}. We created and pilot tested these forms prior to the data extraction. This allowed us to discover potential issues before the study took place, such as ambiguity of questions in the form.

The data extraction was performed by both researchers on each individual study, meaning each study was extracted twice. By doing this, the chance for bias and misinterpretation was reduced. As suggested by Kitchenham \cite{kitchenham2007guidelines}, care was taken to avoid the acceptance of duplicate studies of the same data. In the case of duplication, the most robust study was chosen.

Data from ongoing studies was not included in our data extraction. The primary reason behind this stance is that we were interested in the complete adoption of MDSD and as such a study which is either examining the adoption and is half way through will not help us answer this question.

\begin{table}[ht]
	\centering
	\begin{tabular}{|l|l|l|} 
		\hline
		\textbf{Data Item} & \textbf{Value} & \textbf{Notes}  \\
		\hline
		Data Extractor &  &  \\
		\hline
		Data Checker &  & \\
		\hline
		Study Identifier &  & \\
		\hline
		Industry Domain & & \\
		\hline
		Search Keyword &  & \\
		\hline
		Study Methodology & & \\
		\hline
		Number of Companies Involved &  & \\
		\hline
		Existing Development Practices & & \\
		\hline
		Change Strategies Utilised & & \\
		\hline
		Positive or Negative Conclusion & & \\
		\hline
	\end{tabular}
	\caption{Example Extraction Form}
\end{table}

\subsubsection{Data Synthesis}

As our study included only qualitative studies, a reciprocal translation was was performed. As we focused upon qualitative studies, we were faced with comparing studies consisting of mostly natural language. Consequently, there were some studies which contained differing language for similar concepts but the use of a reciprocal study allowed us to translate the studies so that they could be compared to each other.  

\bibliographystyle{latex8}
\bibliography{latex8}


\Section{Acknowledgements}
John had the main responsibility for sections 1.2, 1.3, 2.0. William had the main responsibility for sections 1.0, 1.1, 2.1, 2.2. The research questions were written together. We have, however, worked on all sections jointly - and all sections have been reviewed by both William and John. 

Parts Section 1.2 have been re-used from previous assignments. Because of this, the paper may show up on Urkund. 



\begin{appendices}
\section{Search Terms}
MDD, MDSD, model-driven software development, model-driven development, MDE, model-driven engineering, model driven engineering, code generation
\end{appendices}

\end{document}

