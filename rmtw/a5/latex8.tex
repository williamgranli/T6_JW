
%
%  $Description: Author guidelines and sample document in LaTeX 2.09$ 
%
%  $Author: John Burchell & William Granli $
%  $Date: 2015/01/21 15:20:59 $
%  $Revision: 1.0
%

\documentclass[10pt,twocolumn]{article} 
\usepackage{latex8}
\usepackage{url}

%\documentstyle[times,art10,twocolumn,latex8]{article}

%------------------------------------------------------------------------- 
% take the % away on next line to produce the final camera-ready version 
\pagestyle{empty}

%------------------------------------------------------------------------- 
\begin{document}



\title{Assignment 5}

% author names and affiliations
\author{John Burchell and William Granli \\
john.a.burchell@gmail.com, william.granli@gmail.com}


\maketitle
\thispagestyle{empty}

%------------------------------------------------------------------------- 



​\Section{Introduction}
Ever since the first computer programs were written, researchers and developers have been working on new programming concepts in order to decrease the level of complexity needed to develop programs. The abstraction level has been increased with help of compilers, object-oriented programming and Unified Modelling Language (UML). The current trend within software engineering (SE) is Model-driven Software Development (MDSD) which uses UML to generate object-oriented code \cite{staron2006adopting}. 

The benefits of MDSD have consistently been reported in academia \cite{staron2006adopting} \cite{volter2013model}. Many large companies such as Ericsson and ABB \cite{staron2006adopting} also use MDSD as a part of their development chain. In industry, however, MDSD struggles to make a (full) (x-wide) (XXX) breakthrough and it is mostly used for niche systems such as embedded and safety-critical systems.

\SubSection{Statement of the Problem}
The selection of development process, programming language and programming paradigm has a high impact on the development speed depending on what the the system requirements are and in which environment the system is developed. Selecting for example MDSD for the wrong system and setting may have a very negative impact on development speed, quality and maintainability. Since there seems to be a consensus in academia that MDSD is an improvement compared traditional software development \cite{staron2006adopting} \cite{volter2013model} and that most large software companies chose not to adopt MDSD, there is a need to methodically investigate the existing literature in which studies of the adoption of MDSD, and advantages and disadvantages of MDSD. 

\SubSection{Existing Literature}
According to literature, changes in companies are commonly implemented following either Software Process Improvement (SPI)\cite{pettersson2008practitioner}\cite{unterkalmsteiner2012evaluation} or change management (CM) strategies. SPI is an attempt to alter an existing company or organisations software process, the main intention of which is to increase product quality, time to market all the while reducing inefficiency and other obstacles for creating good software. 

Similar to SPI is the concept of Evidence Based Software Engineering \cite{dyba2005evidence}. EBSE can be used as support for SPI, as suggested by the creators of the strategy. The main aim of which is to find a technology which is suitable for use and to then implement it in an organisation.

CM on the other hand could be seen as an umbrella for a plethora of theories and strategies. An example of a theory is the concept of innovation values fit by \cite{klein1996challenge} which aims to describe if and when a company should make changes, based upon how a potential change fits with the values currently held by the company.

In 2002 Hall, Rainer and Baddoo \cite{hall2002implementing} published a report wherein they illustrated results of a study which looked at 85 companies throughout the UK. The study focused mostly on how the companies perform SPI and how effective their efforts were. The authors discovered that, while most companies do attempt SPI, many are unsuccessful due to a combined lack of evaluation and funding.

A further, more specific example is a study conducted by Staron \cite{staron2006adopting}. In this study a case study is conducted with two companies wishing or in the process of adopting MDSD. In the findings, Staron provides a set of conditions for adopting MDSD, one being the compatibility of MDSD and existing principles in the company. Staron states that the adoption of this new process must be usable by the previous processes in place \cite{staron2006adopting}. He concludes by stating that this compatibility is especially important for larger companies due to the heavy effort required to implement the change. In a second study \cite{staron2008transitioning}, Staron provides possible reasons for the adoption of MDSD to fail. He suggests that the large investment in time and money, immature technology and organisational resistance are all factors to overcome in the adoption of new processes.

However, there is no existing research that aims to tackle the problem of applying the same change to an organisation that utilises two differing development paradigms. 
\SubSection{Notes4John}
\cite{volter2013model} says that MDSD will: increase development speed, increase quality, improve reusability, improved manageability through abstraction etc. 


\SubSection{Notes4John2}
\cite{mattsson2009linking} talk about the link between MDSD and software architecture. They mention the importance of automatic enforcement of architectural rules and that MDSD is only benefecial if used correctly. 

\SubSection{Motivation for Study}

*5 references only needed for above (re-use from a1)

\Section{Purpose of the Study}
Purpose leading into RQs

\SubSection{Research Questions}
\begin{itemize}
\item \textbf{RQ1:} In which situations should companies expect that changing to MDSD would lead to positive effects? 
\item \textbf{RQ1.1:} In which domain(s) is MDSD the most successful? 
\item \textbf{RQ1.2:} Which software development process(es) has been most successfully used in conjunction with MDSD?
\item \textbf{RQ1.3:} What are the characteristics of the systems of the companies that successfully use MDSD?
\item \textbf{RQ1.4:} Which change strategy has most successful in implementing MDSD? 
\end{itemize}

\begin{itemize}
\item They are probably not worded optimally... Here're some alturnutuves:
\item In which situations should companies expect MDD to be successfully adopted?
\item In which situations should companies expect that the adoption of MDD would lead to a performance boost?
\end{itemize}


\Section{Methodology}

\begin{itemize}
\item Describe and explain the research strategy (i.e. SLR) and investigation methods that will be used for the study. 
\item Motivate your choice of experiment using relevant references. 
\item Search string including keywords and used databases
\item inclusion/exclusion criteria
\item data extraction
\item data synthesis
\item quality assessment
\end{itemize}

kithcneham start

\textbf{Planning the review}
\begin{itemize}
\item Identification of the need for a review (5.1)
Check other SLRs in the field and evaluate them
\item Commissioning a review (5.2)
not needed
\item Specifying the research question(s) (5.3)
this is very important :)
the search process must identify primary studies that address the research questions
the data extraction process must extract the data items needed to answer the questions
the data analysis process must synthesise the data in such a way that the questions can be answered.

read the questions in 5.3.1 to come up with similar ones for readability

\item Developing a review protocol (5.4)

\item Evaluating the review protocol (5.5)
\end{itemize}

\textbf{Conducting the review}
\begin{itemize}
\item Identification of research (6.1)
\item Selection of primary studies (6.2)
\item Study quality assessment (6.3)
\item Data extraction and monitoring (6.4)
\item Data synthesis
\end{itemize}

\textbf{Reporting the review}
\begin{itemize}
\item Specifying dissemination mechanisms (7.1)
\item Formatting the main report (7.2)
\item Evaluating the report (7.3)
\end{itemize} 

\bibliographystyle{latex8}
\bibliography{latex8}


\Section{Acknowledgements}
U w0T m8


\end{document}

