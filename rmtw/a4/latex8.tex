
%
%  $Description: Author guidelines and sample document in LaTeX 2.09$ 
%
%  $Author: John Burchell & William Granli $
%  $Date: 2015/01/21 15:20:59 $
%  $Revision: 1.0
%

\documentclass[10pt,twocolumn]{article} 
\usepackage{latex8}
\usepackage{url}

%\documentstyle[times,art10,twocolumn,latex8]{article}

%------------------------------------------------------------------------- 
% take the % away on next line to produce the final camera-ready version 
\pagestyle{empty}

%------------------------------------------------------------------------- 
\begin{document}



\title{Assignment 4}

% author names and affiliations
\author{John Burchell and William Granli \\
john.a.burchell, william.granli@gmail.com}


\maketitle
\thispagestyle{empty}

%------------------------------------------------------------------------- 

%Deficiencies introduction model


​\Section{Introduction}
Software inspections have since its inception [1] more than
25 years ago spawned quite some interest both from the
research community and industrial practice. The research
includes changes to the inspection process, e.g. [2], support to
the process, e.g. [3] and empirical studies, e.g. [4]. The
suggested improvements include active design reviews [5] and
perspective-based reading [6]. Industry has studied the
benefits of conducting software inspections [7].
The objective of this study is to compare and hence
evaluate how well the use-case based reading performs in
comparison to other methods. The study presents a controlled
experiment where use-case based reading is compared to
checklist-based reading.


\Section{Purpose of the Study}
* Purpose statement (copy paste from the file ‘Experiment\_Design.pdf’ file)

* Research Questions (copy paste from the file ‘Experiment\_Design.pdf’ file)

* Hypothesis (copy paste from the file ‘Experiment\_Design.pdf’ file)

\Section{Methodology}
* Method(s) for analyzing the data (Statistical analysis)

* Motivation of why these statistical tests where used

\Section{Results}
Read \url{https://dl.dropboxusercontent.com/u/2437798/Juristo%20Basics%20of%20Software%20Engineering%20Experimentation.pdf} pp. 133

*Report the results

*Discuss what does the results means, how do we interpret it

\Section{Tests for eff}

%R output start
	F test to compare two variances

data:  uc_eff and cl_eff
F = 1.61, num df = 25, denom df = 25, p-value = 0.2407
alternative hypothesis: true ratio of variances is not equal to 1
95 percent confidence interval:
 0.721893 3.590874
sample estimates:
ratio of variances 
          1.610039 
%R output end


Since the p-value is larger than 0.05, we can deduce that the variances are homogeneous (cite expe) and that the T-test is the optimal approach.

------------------------------------------------------------

%R output start
	Two Sample t-test

data:  uc_eff and cl_eff
t = -1.0786, df = 50, p-value = 0.286
alternative hypothesis: true difference in means is not equal to 0
95 percent confidence interval:
 -3.2585387  0.9816156
sample estimates:
mean of x mean of y 
 6.161538  7.300000 
 %R output end


According to the test, the efficiency of CL is 7.3 while the efficiency of UC is 6.16 and H1 eff is therefore accepted. 

\Section{Tests for rate}

%R output start
	F test to compare two variances

data:  uc_rate and cl_rate
F = 1.7018, num df = 25, denom df = 25, p-value = 0.1907
alternative hypothesis: true ratio of variances is not equal to 1
95 percent confidence interval:
 0.7630304 3.7955018
sample estimates:
ratio of variances 
          1.701788 
%R output end

Since the p-value is larger than 0.05, we can deduce that the variances are homogeneous (cite expe) and that the T-test is the optimal approach.

------------------------------------------------------------

%R output start
	Two Sample t-test

data:  uc_rate and cl_rate
t = -1.1416, df = 50, p-value = 0.2591
alternative hypothesis: true difference in means is not equal to 0
95 percent confidence interval:
 -0.06525100  0.01795869
sample estimates:
mean of x mean of y 
0.1211538 0.1448000 
%R output end

According to the test, the rate of CL is 1.45 while the effectiveness of UC is 0.12 and H1 rate is therefore accepted. 



\bibliographystyle{latex8}
\bibliography{latex8}


\Section{Acknowledgements}
U w0T m8


\end{document}

