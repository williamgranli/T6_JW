
%
%  $Description: Author guidelines and sample document in LaTeX 2.09$ 
%
%  $Author: John Burchell & William Granli $
%  $Date: 2015/01/21 15:20:59 $
%  $Revision: 1.0
%

\documentclass[10pt,twocolumn]{article} 
\usepackage{latex8}
\usepackage{url}
\usepackage{verbatim}

%\documentstyle[times,art10,twocolumn,latex8]{article}

%------------------------------------------------------------------------- 
% take the % away on next line to produce the final camera-ready version 
\pagestyle{empty}

%------------------------------------------------------------------------- 
\begin{document}



\title{Assignment 4}

% author names and affiliations
\author{John Burchell and William Granli \\
john.a.burchell, william.granli@gmail.com}


\maketitle
\thispagestyle{empty}

%------------------------------------------------------------------------- 


\Section{Methodology}
* Method(s) for analyzing the data (Statistical analysis)

* Motivation of why these statistical tests where used

\Section{Results}

*Report the results

*Discuss what does the results means, how do we interpret it

\Section{Tests for eff}

\verbatiminput{f_eff.txt}

Since the p-value is larger than 0.05, we can deduce that the variances are homogeneous (cite expe) and that the T-test is the optimal approach.

\verbatiminput{t_eff.txt}

According to the test, the efficiency of CL is 7.3 while the efficiency of UC is 6.16 and H0 eff is therefore rejected. 

\Section{Tests for rate}

\verbatiminput{f_rate.txt}



Since the p-value is larger than 0.05, we can deduce that the variances are homogeneous (cite expe) and that the T-test is the optimal approach.


\verbatiminput{t_rate.txt}

According to the test, the rate of CL is 1.45 while the effectiveness of UC is 0.12 and h0 rate is therefore rejected. 

\Section{Tests for #3}

%%R output
	Chi-squared test for given probabilities

data:  ct
X-squared = 5.4847, df = 2, p-value = 0.06442

Since our significance level is 0.05 and p > 0.05, there is no significant difference and h1 is rejected. 
%%R output


Chi-square test requirements[edit]
Quantitative data.
One or more categories.
Independent observations.
Adequate sample size (at least 10).
Simple random sample.
Data in frequency form.
All observations must be used.
\bibliographystyle{latex8}
\bibliography{latex8}


\Section{Acknowledgements}
U w0T m8


\end{document}

