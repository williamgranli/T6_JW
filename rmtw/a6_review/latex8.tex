
%
%  $Description: Author guidelines and sample document in LaTeX 2.09$ 
%
%  $Author: John Burchell & William Granli $
%  $Date: 2015/01/21 15:20:59 $
%  $Revision: 1.0
%

\documentclass[times, 10pt,twocolumn]{article} 
\usepackage{latex8}
\usepackage{url}
\usepackage{tabularx}
\usepackage{appendix}
\usepackage{pbox}

%\documentstyle[times,art10,twocolumn,latex8]{article}

%------------------------------------------------------------------------- 
% take the % away on next line to produce the final camera-ready version 
\pagestyle{empty}

%------------------------------------------------------------------------- 
\begin{document}

\title{Review of Group 10 by Group 1}

% author names and affiliations
\author{John Burchell and William Granli \\
john.a.burchell@gmail.com, william.granli@gmail.com}



\maketitle
\thispagestyle{empty}

\section{Title}

The title could be improved. The current one doesn't describe the connection you will make between working climate and gender aspects. An example would be: ``Comparing and Contrasting the Perception of Work Climate Between Sexes". We would use the word work instead of working and the  word sex instead of gender. Gender is usually used in grammar and sex when talking about men and women. 

\section{Introduction}
The first sentence of the paper is incorrectly cited. The paper mentions the engineering profession in general but you write ``computer science". The paper also only has 1 cite, so you should probably reconsider the quality of it. 

There is no clear problem statement. (We assume that you 2nd paragraph is the problem statement). Make it more clear and define an explicit problem.

You should also consider adding a bit more references to back up your statements and to show that there has been more than 1 paper in an area. (Add several references to 1 sentence). If you can't find more papers, state that in your study so that it is clear to the reader. 

\section{Related Work}
The stuff about schools/education and pair programming seems a bit unrelated to your topic. If it is relevant, make the connection more clear.

\section{Purpose}
Very good! Clear and explains what it should!

\section{Case Company Description}
The first sentence gives the impression that your study is something negative for the company. You could reword this to ``The company involved has wished to remain anonymous due to internal policies". 

A good general intro to the company. You should, however, explain why this case is well-suited to perform gender studies at. (For example, the 25\% thing is a good reason, so that could be mentioned a bit more clearly). You should also explain what criteria you used for selecting the case company and how they were sampled. 

\section{Methodology}
``We have created some key area": You need to argue why these are the most optimal to answer your research questions. Why were they chosen? Please also explain what all the areas mean in this context. ``Knowledge'', for example is quite ambiguous. 

Is there a clear difference between sub-questions a, b and c? It seems like a should bring up everything that b and c answers. The sub-questions should also be re-written so that they are full and standalone questions. 

The paragraph after the research questions: The arguments  are a bit hard to follow, consider improving this paragraph. The next section is very good!

\subsection{Data Collection}
You use the word convenient, and it sounds like you are doing things to make it easy for yourself. Try to reword this. (We assume it wasn't your intention). 

Try to explain why these data collection methods are good to answer your research questions. E.g. why are interviews needed to answer your specific questions?

Otherwise the section is good!


\subsection{Data Analysis}
One of your best sections, except the last part. You mention your company's initiative to increase the number of female managers etc. This probably reduces the validity of your study since it introduces bias in the data you are analysing, so keep this in mind. 

\subsection{Validity Threats}
Have a look at the [Case] paper. It has a really good explanation on how to write these sections for case studies. You should probably explain how you would counteract bias towards publishing results that are positive for one of the genders. (For example, that there is bias in something which would lead to positive results about men). Since this is a central part of your study. Also explain how you have counteracted that bias. 

The final validity threat (about career and political correctness) is really good!

\end{document}

