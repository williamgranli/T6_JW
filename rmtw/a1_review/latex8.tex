
%
%  $Description: Author guidelines and sample document in LaTeX 2.09$ 
%
%  $Author: John Burchell & William Granli $
%  $Date: 2015/01/21 15:20:59 $
%  $Revision: 1.0
%

\documentclass[times, 10pt,twocolumn]{IEEEtran} 
\usepackage{latex8}
\usepackage{times}

%\documentstyle[times,art10,twocolumn,latex8]{article}

%------------------------------------------------------------------------- 
% take the % away on next line to produce the final camera-ready version 
\pagestyle{empty}

%------------------------------------------------------------------------- 
\begin{document}

\title{Review of Group 8 by Group 1}

% author names and affiliations
\author{\IEEEauthorblockN{
  John Burchell\IEEEauthorrefmark{1} and
  William Granli\IEEEauthorrefmark{1}}

  \IEEEauthorblockA{\IEEEauthorrefmark{1}john.a.burchell, william.granli@gmail.com}

}


\maketitle
\thispagestyle{empty}





%------------------------------------------------------------------------- 


\Section{Introduction}
The literature review is well-written, extensive and seems to cover a lot of surrounding studies. It is good that a paper supporting the importance of personal fulfilment has been cited, since your paper almost stands and falls by this. One improvement for the Related Work section could be to write it as one text where you content drive the text instead of listing the papers one by one. This way you would get a more condensed text which is constructed around your paper. 

\Section{Purpose Statement}
The problem statement is very clear. It would be good to state a couple of incentives for the industry, however. What would your paper bring to the industry? The benefits to academia have been mentioned in a good way. The sentence ``The purpose of this survey is to collect data about'' could probably be elaborated a bit. Explain what you will do with the data, for example. 
\\

The definition of expertise is a bit too vague and ambiguous, in our opinion. You can probably replace expertise with years of experience and achieve the same results, since there should be a strong correlation between the two. The same goes for partner's skill level. It is a bit unclear from the paper what partner's skill level means. It can probably be clarified that partner's skill level is a 1-5 rating made by the other partner. 


\Section{Research Question}
The research questions are yes-no questions. The main research question could be rephrased to, for example: ``How does pair programming affect/correlate to personal fulfilment of a developer?''. The sub-questions could be rephrased in similar style. 
\\

The sub-questions are good, but it is quite unclear what your intent with them is. The difference between expertise and skill-level isn't really clear and it should probably be emphasised that you want to measure the difference between ``actual skill-level'' and ``skill-level as perceived by your partner''. 


\Section{Suitable Research Design}
The description of your population is clear and well-researched. It is good that you spent time to research how you find the most meaningful population, by for example leaving out inexperienced developers. One improvement to your research design section would be to, in the introduction, briefly outline your research process and describe which stages it includes and it what order they will be carried out. 
\\

In the results analysis section, you should probably mention which type of analysis you will apply. It is probably descriptive analysis in your case. You should also motivate why a more extensive approach was not ensued. 
\\

The identified limitations are good. One suggestion we have would be to research how different types of pair-programming impact personal fulfilment. You mentioned the driver and navigator style. Maybe there are other styles that would get widely different results in the survey. If this seems feasible, a ``type of pair-programming'' independent variable could be added to your survey. 


\Section{Other}
Below sentence does not make sense: \\
``The purpose of this survey study is to assess the impact of pair programming on personal fulfilment, controlling for expertise of developers with pair programming at mid sized software development organisations. ''






\end{document}

