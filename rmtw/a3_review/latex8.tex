
%
%  $Description: Author guidelines and sample document in LaTeX 2.09$ 
%
%  $Author: John Burchell & William Granli $
%  $Date: 2015/01/21 15:20:59 $
%  $Revision: 1.0
%

\documentclass[times, 10pt,twocolumn]{article} 
\usepackage{latex8}
\usepackage{times}

%\documentstyle[times,art10,twocolumn,latex8]{article}

%------------------------------------------------------------------------- 
% take the % away on next line to produce the final camera-ready version 
\pagestyle{empty}

%------------------------------------------------------------------------- 
\begin{document}

\title{Review of Group 8 by Group 1}

% author names and affiliations
\author{John Burchell and William Granli \\
john.a.burchell@gmail.com, william.granli@gmail.com}



\maketitle
\thispagestyle{empty}





%------------------------------------------------------------------------- 


\Section{Introduction \& Related Work}
The introduction was once again very well written. Good use of italicising the new and key terms meant that the phenomena and key ideas are easily identifiable. 

``consider the personal fulfilment factors related to qualities such as personal fulfilment factors related to qualities such as engagement,'' Reads very strangely, I presume this was an oversight? The first paragraph of the related work section also sounds a little strange. The related work does a great job of giving a good background and introduction into. It seems to have been very well researched and there are convincing arguments for inclusions and exclusions of concepts.

\Section{Purpose Statement}

The purpose statement is clear but with one minor issue, there is no definition of expertise. What do you mean by expertise in this context, years of experience? How ``good'' a developer is, if so, compared to what or who? What their discipline is? Just a slight clarification would be nice. On the other hand, it was good to include a reason why you were ignoring personal ratings of expertise.

\Section{Research Question}
Research Questions make sense and are logical compared to the purpose statement and the sub-questions presented.
RQ2 - Is that what you mean by expertise? A combination of general programming ability combined with pair programming experience?

\Section{Methodology}
page 2, line 4 ``the purpose the study''.
The methodology is well written and provides a good argument for why a case study is being conducted. There was some good information regarding the agreement to keep the results and participants of the study anonymous for various reasons.

\subsection*{Data Collection}
Good information here especially in relation to how the studies are going to be conducted. Especially, the note about how the subjects will be studied twice and the reasons for this. Perhaps consider mentioning that the study is exploratory a little sooner in the paper?

\subsection*{Data Analysis}
The data analysis was very thorough, with special attention being paid to the codification and to how the data will be used. In particular, the suggested next steps by means of finding and support hypothesis for future research.

slight typo here: ``to protect the subject’s identify '', end of second paragraph.

\subsection*{Interview Structure}
This section gave a nice overview of the interview structure which is supported by the index and other tables of information. The inclusion of the immediate feedback loop for the interview questions was a good point to illustrate.

One problem there could be with the questions is that terms such as engagement might be a bit ambiguous to some people, does it refer to how engaged they feel in the task? 

However, the sub questions and open ended questions are written very well, they clearly state what the researchers should be gathering from their subjects by explicitly stating it under the question.

\Section{Validity threats}
Nice and concise with no major flaws other than the fact that you didn't mention how the person (Your company contact) who will chose the participants of the interviews could be a potential validity threat, i.e. he might pick those whom he feels are the best pairs, or worst pairs e.t.c.

\Section{Other}
What would you look out for when observing? Is there anything in particular? If so, it might be nice to see these notes, or even a ``suggested form'' that the researchers might be using to log these findings in.


\end{document}

