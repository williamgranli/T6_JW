
%
%  $Description: Author guidelines and sample document in LaTeX 2.09$ 
%
%  $Author: ienne $
%  $Date: 1995/09/15 15:20:59 $
%  $Revision: 1.4 $
%

\documentclass[times, 10pt,twocolumn]{IEEEtran} 
\usepackage{latex8}
\usepackage{times}

%\documentstyle[times,art10,twocolumn,latex8]{article}

%------------------------------------------------------------------------- 
% take the % away on next line to produce the final camera-ready version 
\pagestyle{empty}

%------------------------------------------------------------------------- 
\begin{document}

\title{Title of the document here}

% author names and affiliations
\author{\IEEEauthorblockN{
  John Burchell\IEEEauthorrefmark{1} and
  William Granli\IEEEauthorrefmark{1}}

  \IEEEauthorblockA{\IEEEauthorrefmark{1}john.a.burchell, william.granli@gmail.com}

}
\maketitle
\thispagestyle{empty}

\begin{abstract}
   The ABSTRACT is to be in fully-justified italicized text, at the top 
   of the left-hand column, below the author and affiliation 
   information. Use the word ``Abstract'' as the title, in 12-point 
   Times, boldface type, centered relative to the column, initially 
   capitalized. The abstract is to be in 10-point, single-spaced type. 
   The abstract may be up to 3 inches (7.62 cm) long. Leave two blank 
   lines after the Abstract, then begin the main text. 
\end{abstract}



%------------------------------------------------------------------------- 
\Section{Introduction}

1. Statement of the problem
  Measuring the inherent readability of various programming languages. 
  \newline \newline
2. Existing literature
  No previous studies have compared the readability of different programming languages. There are however existing papers that explore the concept of readability. \cite{buse2010learning} talks about readability and presents a metric system to automatically measure readability. \cite{posnett2011simpler} build on \cite{buse2010learning}'s concept to provide a simpler and more accessible metric system. \cite{hansen2013makes} focus on what makes code hard to read and presents those factors. 
  \newline \newline
3. Why is it important?
\cite{aggarwal2002integrated} talks about code maintainability and mentions readability as a factor related to maintainability. 

\cite{elshoff1982improving} also talks about readability and how it impacts maintainability. \cite{elshoff1982improving} brings up specific coding standards/conventions/practices to show differences in readability. 


\textbf{Industry}: To compare code of different language to be able to prioritize them. \\
\textbf{Academia}: To conduct more studies on the area which compare different languages. \\
To determine which languages are most suited for academic teaching. (When other attributes of programming languages are not constraints).


%------------------------------------------------------------------------- 
\Section{Purpose of the Study}

*Purpose statement (concise  and clear)  leading into  RQ(s)
 -i.e.  describe  the purpose of  your  survey/questionnaire

%------------------------------------------------------------------------- 
\SubSection{Research Question}
*State a clear and concise main  research  question  that  will  be  investigated
*State another one or  two sub research  questions related to  the main  RQ.


%------------------------------------------------------------------------- 
\SubSection{Methodology}

* Describe  and explain the research  strategy (i.e.  Survey/Questionnaire)
and investigation methods that  will  be  used  for the study.  Motivate  your  
choice  of  survey/questionnaire  using relevant  references.
* Describe  data  collection  procedures, sampling  method(s),  number  of  
subjects, response  rate  etc.
* Create  questions for your  survey  (include  the questions in  your  report)
* Describe  the population  and size  of  population
* Describe  and motive  method  for analyzing the data
* Discuss validity  threats

%------------------------------------------------------------------------- 


\bibliographystyle{assignment_1}
\bibliography{assignment_1}

\end{document}

