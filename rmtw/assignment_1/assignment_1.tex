
%
%  $Description: Author guidelines and sample document in LaTeX 2.09$ 
%
%  $Author: ienne $
%  $Date: 1995/09/15 15:20:59 $
%  $Revision: 1.4 $
%

\documentclass[times, 10pt,twocolumn]{IEEEtran} 
\usepackage{latex8}
\usepackage{times}




%\documentstyle[times,art10,twocolumn,latex8]{article}

%------------------------------------------------------------------------- 
% take the % away on next line to produce the final camera-ready version 
\pagestyle{empty}

%------------------------------------------------------------------------- 
\begin{document}

\title{Title of the document here}

% author names and affiliations
\author{\IEEEauthorblockN{
  John Burchell\IEEEauthorrefmark{1} and
  William Granli\IEEEauthorrefmark{1}}

  \IEEEauthorblockA{\IEEEauthorrefmark{1}john.a.burchell, william.granli@gmail.com}

}
\maketitle
\thispagestyle{empty}

\begin{abstract}
   The ABSTRACT is to be in fully-justified italicized text, at the top 
   of the left-hand column, below the author and affiliation 
   information. Use the word ``Abstract'' as the title, in 12-point 
   Times, boldface type, centered relative to the column, initially 
   capitalized. The abstract is to be in 10-point, single-spaced type. 
   The abstract may be up to 3 inches (7.62 cm) long. Leave two blank 
   lines after the Abstract, then begin the main text. 
\end{abstract}



%------------------------------------------------------------------------- 
\Section{Introduction}
%Problem and definition
Comparisons between programming languages are difficult to perform, due to inherent differences between languages, such as syntax. One aspect of code is its readability, which in turn is a determining factor of a piece of software's quality and maintainability \cite{aggarwal2002integrated, elshoff1982improving}. Buse \& Westley define readability as ``a property that influences how easily a given piece of code can be read and understood'' \cite{buse2010learning}. 
\newline

%%Existing literature
Existing studies in the field focus on readability within of different coding styles for a set language, which can be seen in for example \cite{buse2010learning}. Buse \& Westley introduce a metrics system for measuring the readability of code, and Java is the language of choice in the study \cite{buse2010learning}. Buse \& Westley also discuss readability and its impact on programming language design, but they do not explore the readability differences between languages \cite{buse2010learning}. Posnett et al. \cite{posnett2011simpler} have further developed Buse \& Westley's metrics model to provide a simpler and more practical model which builds on size metrics as well as Halstead metrics \cite{halstead1977elements}. Hanen et al. have presented a study in the field which aims to identify details in code which makes it hard to read and the main focus is put on syntactical elements such as indentation or curly brackets \cite{hansen2013makes}. Even though direct comparisons between languages were not presented in the study \cite{hansen2013makes}, conclusions related to this can be drawn thanks to the fact that these syntactical elements only exist in some languages. 
\newline

The importance of code readability may be overlooked, since it does not directly affect runtime performance. Aggarwal et al., however claim that code's readability impacts it's maintainability \cite{aggarwal2002integrated}. This claim is also supported by \cite{elshoff1982improving} et al. who also introduce examples of certain coding practices that are more readable and maintainable than others \cite{elshoff1982improving}.


%------------------------------------------------------------------------- 
\Section{Purpose of the Study}

%Introduction, specifying why this is urgent, needed etc from the notes above

Both industry and academia face the problem of maintainability. As stated previously, many times the readability of code affects the codes maintainability. This study could help some in industry to pick languages for certain tasks for maintainability reasons alone. Further studies could be conducted by to determine if there is a correlation between the popularity of projects and the readability of their languages. Furthermore, academia would be interested in which languages are more popular and useful for learning. By further investigating the design of the programming languages themselves, future languages which are developed could build upon the success of those which came before them.
\newline

The purpose of this survey study is therefore to test what are the differences in readability between five common and well known programming languages. With the aim of developing a metric to describe the readability of a given program, which takes into account the inherent readability of the language in which it is written.
\newline

The purpose of this survey study is therefore to find what are the differences in readability between five common and well known programming languages. With the aim of developing a metric to describe the readability of a given program, which takes into account the inherent readability of the language in which it is written.
\newline

The Independent variables will be defined as the experience, occupation, preferred language, least preferred language and the working language of the participant. The dependent variable in this example will be the readability of the language.
\newline

The survey data will be collected from professionals, teachers, students and hobbyists via a series of on-line surveys. The respondents will then be asked to review a series of code snippets from popular languages and will be subsequently asked to rate their readability. 
\newline

%------------------------------------------------------------------------- 
\SubSection{Research Question}
This study has been designed with one main research question in mind:
\begin{itemize}
\item \textit{RQ1:} What are the measurable differences in readability between programming languages?
\end{itemize}


The results of RQ1 will then be used to answer the following two sub research questions. 
\begin{itemize}
\item \textit{SRQ1}: How do the measurable differences between languages correlate to their use in academia for teaching?
\item \textit{SRQ2}: How do the measurable differences between languages correlate to the popularity of the language?
\end{itemize}

\begin{tabular}{| l | r | c |}
	\hline
	\textbf{Variable Name} & \textbf{Research Question} & \textbf{Item on Survey} \\ \hline
	Occupation & RQ1, SQ1 & \\ \hline
	Experience & RQ1, SQ1 & \\ \hline
	Preferred Language & RQ1, SQ2 & \\ \hline
	Least Preferred Language & RQ1, SQ2& \\  \hline
	Working Language & RQ1, SQ2 & \\ 
	\hline
	\end{tabular}
%------------------------------------------------------------------------- 
\Section{Methodology}
The main purpose of this study is to discover differences in the perceived readability of code. The study will be carried out in three phases. In phase 1, we will identify the target population. In phase 2, the survey will be designed and carried out. In phase 3, the results will be analysed. The investigation method will be in the form of a questionnaire. The main argument for the questionnaire is that the survey must be distributed to a large amount of people. A quantitative is also well-suited for this study since there are no existing publications that explore readability differences between languages. 

\SubSection{Identifying the Population}
As mentioned before, the goal of this study is to identify the inherent differences in readability of programming languages. To research this, a broad population is required, since there may be strong correlation between the perception of readability and the previously defined independent variables. As such, a research survey will be conducted. This reasoning is reinforced by Easterbrook et al. \cite{easterbrook2008selecting} who state ``Survey research is used to identify the characteristics of a broad population of individuals''. 

* Describe  and explain the research  strategy (i.e.  Survey/Questionnaire)
and investigation methods that  will  be  used  for the study.  Motivate  your  
choice  of  survey/questionnaire  using relevant  references.
\newline


\SubSection{Questionnaire}
Blabla

\SubSection{Results Analysis}




We will use a survey/questionnaire to collect the data.
Sampling - ID a small subset of the entire population
\newline

* Describe  data  collection  procedures, sampling  method(s),  number  of  
subjects, response  rate  etc.
\newline

Random sampling from 200 people(guess). Sample 10\%, collect online via survey, Response rate of 15/20 (75\%)
\newline

Sampling bias - What we might face, how we overcome. State which threats and how they were minimised
\newline

* Create  questions for your  survey  (include  the questions in  your  report)
%Table goes here
\newline

* Describe  the population  and size  of  population
\newline
The population will be software developers in the SEM program. This includes teachers, students, professionals and hobbyists. Lets say the sample is 20 persons from the whole of SEM, 1st, 2nd, 3rd and staff.
\newline
\cite{easterbrook2008selecting} - ``A major challenge in survey research is to control for sampling bias'' \& `` the respondents to 
the survey may not be representative of the target population. ''
\newline

* Describe  and motive  method  for analyzing the data
\newline

* Discuss validity  threats
  Sampling Bias ?

\SubSection{Literature Review}
The first phase will include a literature review. The main goal is to review previously published papers which are related to code readability. The existing studies will serve as a foundation for our study, as the goal of our study is not to further develop or research code readability, but compare readability between languages (using academia's existing definition of readability). The review will also serve as a base for the second phase, where the target population will be decided. 
\newline

In phase 1, an initial literature review will be conducted. The aim is to cover previously published studies to be able to provide a study which is purposeful for academia.


%------------------------------------------------------------------------- 

\textbf{Industry}: To compare code of different language to be able to prioritize them. \\
Further studies can be conducted to see if there is a correlation between popularity of projects and the readability of the langauage(s) used in the project. \\
\textbf{Academia}: To conduct more studies on the area which compare different languages. \\
To determine which languages are most suited for academic teaching. (When other attributes of programming languages are not constraints).



\bibliographystyle{assignment_1}
\bibliography{assignment_1}

\end{document}

