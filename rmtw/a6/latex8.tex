%
%  $Description: Author guidelines and sample document in LaTeX 2.09$ 
%
%  $Author: John Burchell & William Granli $
%  $Date: 2015/01/21 15:20:59 $
%  $Revision: 1.0
%

\documentclass[10pt,twocolumn]{article} 
\usepackage{latex8}
%\documentstyle[times,art10,twocolumn,latex8]{article}

%------------------------------------------------------------------------- 
% take the % away on next line to produce the final camera-ready version 
\pagestyle{empty}

%------------------------------------------------------------------------- 
\begin{document}



\title{A Comparison of Use-Case Based and Checklist-
Based Reading: A Controlled Experiment}

% author names and affiliations
\author{John Burchell and William Granli \\
john.a.burchell, william.granli@gmail.com}


\maketitle
\thispagestyle{empty}

%------------------------------------------------------------------------- 

\section{Introduction}
%API Design (and a little bit of ecosystems, and a bit of problem statement)
An API is an interface defined in a programming language (in code) which can be accessed programatically by other software. (!!!Re-word this sentence a bit) There is a consensus in that modifying APIs generally should be avoided if possible \cite{google_talk} \cite{mcdonnell2013empirical} \cite{robbes2012developers} \cite{henning2007api} \cite{robbes2012developers} since it will require the users of the API to update the code that is accessing the API (!!!re-word), and thus cause a disruption in the software ecosystem \cite{messerschmitt2005software} that the API's platform is a part of. Most commonly, already deployed APIs are modified due to code refactoring being made. \cite{dig2005role} \cite{xing2006refactoring}. 
%Software Evolution 
In the discipline of Software Evolution, such updates to software are studied from an evolutionary standpoint. Software can be updated for different reasons and these motives can be grouped into corrective, adaptive, perfective and preventive changes \cite{lientz1980software} \cite{iso}. 

%Gap in Literature
Existing studies that explore the intent behind API modifications exist \cite{hou2011exploring}. There are however no studies that attempt to map the motives behind changes to business needs or existing software evolution theories such as the four categories of maintenance \cite{lientz1980software}. The area of programming language APIs is well-explored \cite{hou2011exploring} \cite{shi2011empirical}, but few studies that explore platform APIs exist \cite{robbes2012developers}. No studies that explore the motives behind or the effects of API evolution have previously been performed on embedded platform APIs. 

%Purpose (and a bit of problem statement)
Because of the negative implications that modifications to to APIs cause for the API users, it is important to understand why changes to APIs happen. If most common pitfalls are known before the development of new APIs are known, these mistakes can mitigated. Before the industry can make use of this information, the proposed common pitfalls must also be validated. 


\subsection{Research Questions}
\begin{itemize}
\item RQ1: What drives API evolution? 
\item RQ1.1: To what extent can we reverse engineer API change decisions?
\item SRQ1.2: Are changes/improvements to APIs made for good reasons? (Compare why the changes were made to best practises for why you should change APIs)
\end{itemize}


\section{Related Work}

\subsection{API Design}

\subsection{Software Evolution}



\section{Case Company Description}



\section{Methodology}
Brief intro, saying we will do a case study and what data collection methods will be used. (Also citing the paper from where we steal the case study process... höst).

\subsection{Case Study Design}


\subsection{Preparation for Data Collection}

\subsection{Data Gathering}

\subsection{Data Analysis}

\subsection{Threats to Validity}



\bibliographystyle{latex8}
\bibliography{latex8}


\Section{Acknowledgement}

\noindent
\end{document}

