%
%  $Description: Author guidelines and sample document in LaTeX 2.09$ 
%
%  $Author: John Burchell & William Granli $
%  $Date: 2015/01/21 15:20:59 $
%  $Revision: 1.0
%

\documentclass[10pt,twocolumn]{article} 
\usepackage{latex8}
%\documentstyle[times,art10,twocolumn,latex8]{article}

%------------------------------------------------------------------------- 
% take the % away on next line to produce the final camera-ready version 
\pagestyle{empty}

%------------------------------------------------------------------------- 
\begin{document}



\title{A Comparison of Use-Case Based and Checklist-
Based Reading: A Controlled Experiment}

% author names and affiliations
\author{John Burchell and William Granli \\
john.a.burchell, william.granli@gmail.com}


\maketitle
\thispagestyle{empty}

%------------------------------------------------------------------------- 

\section{Introduction}
%Literature intro
%A bit unclear what to write here - check with Branson
Explain what APIs are. 
Explain how and that APIs change. (Link it to the software evolution in a nice way). 


\subsection{Problem Statement}
1. 

\subsection{Purpose of the Study}
Academia: 
The results of what *actually* (i.e. our study) affects APIs could be compared to what literature says should affect design (i.e. best practices). 

Industry: 
If a company can study their API's evolution and be able to track changes and design decisions, it can give them more knowledge to further improve and develop their API. 

If companies can identify changes, they can also identify what *was* bad.



Changing APIs is usually not a good thing. If you can identify what changes APIs you can spend more time on that in development to avoid too much change in the future. 


\subsection{Research Questions}
RQ1: What drives API evolution? 
SRQ1: To what extent can we reverse engineer API change decisions?
SRQ2: 

\section{Related Work}

\subsection{API Design}

\subsection{Software Evolution}

\subsection{Architecture / Reverse Engineering (or tracing previously made decisions)}

\section{Case Company Description}



\section{Methodology}
Brief intro, saying we will do a case study and what data collection methods will be used. (Also citing the paper from where we steal the case study process... höst).

\subsection{Case Study Design}


\subsection{Preparation for Data Collection}

\subsection{Data Gathering}

\subsection{Data Analysis}

\subsection{Threats to Validity}



\bibliographystyle{latex8}
\bibliography{latex8}


\Section{Acknowledgement}

\noindent
\end{document}

