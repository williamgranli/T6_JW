%
%  $Description: Author guidelines and sample document in LaTeX 2.09$ 
%
%  $Author: John Burchell & William Granli $
%  $Date: 2015/01/21 15:20:59 $
%  $Revision: 1.0
%

\documentclass[10pt,twocolumn]{article} 
\usepackage{latex8}
%\documentstyle[times,art10,twocolumn,latex8]{article}

%------------------------------------------------------------------------- 
% take the % away on next line to produce the final camera-ready version 
\pagestyle{empty}

%------------------------------------------------------------------------- 
\begin{document}



\title{A Comparison of Use-Case Based and Checklist-
Based Reading: A Controlled Experiment}

% author names and affiliations
\author{John Burchell and William Granli \\
john.a.burchell, william.granli@gmail.com}


\maketitle
\thispagestyle{empty}

%------------------------------------------------------------------------- 

\section{Introduction}
%API Design (and a little bit of ecosystems, and a bit of problem statement)
As the software industry and the open source movement steadily grow, the number of public APIs is increasing. APIs can improve the development speed \cite{stylos2006comparing}, contribute to higher quality software \cite{stylos2006comparing} and increase the reusability of software \cite{afonso2012evaluating}.  There is a consensus in that modifying APIs generally should be avoided if possible \cite{google_talk} \cite{mcdonnell2013empirical} \cite{robbes2012developers} \cite{henning2007api} \cite{robbes2012developers} since it will require the users of the API to update the code that is accessing the API, thus causing a disruption in the software ecosystem \cite{messerschmitt2005software} that the API's platform is a part of. Most commonly, already deployed APIs are modified due to code refactoring being made \cite{dig2005role} \cite{xing2006refactoring}. 
%Software Evolution 
In the discipline of Software Evolution, such updates to software are studied from an evolutionary standpoint. Software can be updated for different reasons and these motives can be grouped into corrective, adaptive, perfective and preventive changes \cite{lientz1980software} \cite{iso}. 

%Gap in Literature
Existing studies that explore the intent behind API modifications exist \cite{hou2011exploring}. There are however no studies that attempt to map the motives behind changes to business needs or existing software evolution theories such as the four categories of maintenance \cite{lientz1980software}. The area of programming language APIs is well-explored \cite{hou2011exploring} \cite{shi2011empirical}, but few studies that explore platform APIs exist \cite{robbes2012developers}. To the best of our knowledge, no studies that explore the motives behind or the effects of API evolution have previously been performed on embedded platform APIs. 

%Purpose (and a bit of problem statement)
Because of the negative implications that modifications to to APIs cause for the API users, it is important to understand why changes to APIs happen. If most common pitfalls are known before the development of new APIs begin, these mistakes could be mitigated. Before the industry can make use of this information, we must also critically assess to what degree we trace such motives. To achieve this, we will perform a case study which aims to investigate the underlying factors behind what drives changes to APIs. This will be done by empirically analysing the changes between two versions of an API. Subsequently, the designers of the API will be interviewed and to verify (!!!does this imply that our findings will be accurate?) the results of our analysis. 

\subsection{Research Questions}
\begin{itemize}
\item RQ1: What drives API evolution? 
\item RQ1.1: To what extent can we reverse engineer API change decisions?
% \item SRQ1.2: Are changes/improvements to APIs made for good reasons? (Compare why the changes were made to best practises for why you should change APIs)
\end{itemize}


\section{Related Work}

\subsection{API Design}

API design is notoriously difficult, a myriad of design and performance decisions must be taken into consideration when creating APIs \cite{bloch2008effective},\cite{afonso2012evaluating}, \cite{stylos2006comparing}. Some studies have created a generalisable usability study with the aim of analysing the design decisions taken when creating an API \cite{}. Another empirical study, which involved working with 25 programmers, highlighted API design problems such as assigning names to API features and difficulty in naming types \cite{shi2011empirical}. The study also found that in order to create a usable API, one must ensure that the API is understandable via good documentation, it’s not too abstract, has high reusability and that it is easy to learn \cite{shi2011empirical}. Both studies also highlighted that API designers must ensure that, when using design patterns, such as the concrete factory pattern, that the intent of the API is clear to the user \cite{shi2011empirical},\cite{stylos2006comparing}.

A further study involved combining semiotic and cognitive methods to evaluate APIs as a means of facilitating communication, via an artefact, between programmers and designers. \cite{afonso2012evaluating}. However, these studies do not highlight the actual reasons behind changes of APIs.

\subsection{API Change}
Previous studies that have attempted to identify the reason behind API changes have primarily focused APIs that are a part of large programming languages, such as Java and Smalltalk \cite{robbes2012developers}, \cite{shi2011empirical}, \cite{hou2011exploring}. The intent of the study was to uncover the intents behind the changes made in the AWT/Swing Java libraries. The study found that the use of a strong architecture was vital in ensuring a smooth evolution of the API \cite{hou2011exploring}.

However, when APIs are changed, problems tend to occur. One study investigated three frameworks and one library in which the study found that 80\% of re-factoring changes to APIs negatively affected existing applications. Re-factorings are not the only changes to APIs that can impact existing projects negatively. Another study was conducted regarding the changes, or, ripple effect that changes in APIs have on an entire ecosystem. This particular empirical study found that 14\% of non-trivial API deprecations caused errors in at least one project, with the worst case of 79 projects being affected \cite{robbes2012developers}.

Furthermore, a study of the impact of reliable APIs was undertaken in the Android ecosystem \cite{mcdonnell2013empirical}. The study aimed to discover if APIs with high rates of faults affected the success of an application. The study found that the 50 least popular applications had APIs which were 500\% more error prone, ultimately concluding that there is indeed a correlation between API stability and application success \cite{mcdonnell2013empirical}. 

\section{Case Company Description}



\section{Methodology}
Brief intro, saying we will do a case study and what data collection methods will be used. (Also citing the paper from where we steal the case study process... höst).

\subsection{Case Study Design}


\subsection{Preparation for Data Collection}

\subsection{Data Gathering}

\subsection{Data Analysis}

\subsection{Threats to Validity}



\bibliographystyle{latex8}
\bibliography{latex8}


\Section{Acknowledgement}

\noindent
\end{document}

