%
%  $Description: Author guidelines and sample document in LaTeX 2.09$ 
%
%  $Author: John Burchell & William Granli $
%  $Date: 2015/01/21 15:20:59 $
%  $Revision: 1.0
%

\documentclass[10pt,twocolumn]{article} 
\usepackage{latex8}
%\documentstyle[times,art10,twocolumn,latex8]{article}

%------------------------------------------------------------------------- 
% take the % away on next line to produce the final camera-ready version 
\pagestyle{empty}

%------------------------------------------------------------------------- 
\begin{document}



\title{A Comparison of Use-Case Based and Checklist-
Based Reading: A Controlled Experiment}

% author names and affiliations
\author{John Burchell and William Granli \\
john.a.burchell, william.granli@gmail.com}


\maketitle
\thispagestyle{empty}

%------------------------------------------------------------------------- 

\section{Introduction}
%API Design
An API is an interface defined in a programming language (in code) which can be accessed programatically by other software. (!!!Re-word this sentence a bit) %Add some more filler stuff here

%Problem statement
There is consensus in that modifying APIs generally should be avoided if possible \cite{google_talk} \cite{mcdonnell2013empirical} \cite{robbes2012developers} \cite{henning2007api} \cite{robbes2012developers} since it may will the users of the API to update the code that is accessing the API. (!!!re-word) The most common reason for modifying an API after it has been deployed (!!! word?) is refactoring \cite{dig2005role} \cite{xing2006refactoring}. %refactoring isn't the reason... it's the process. look up why refactoring is done or change wording.


%Software Evolution 
In the discipline of Software Evolution, such updates to applications are studied from an evolutionary standpoint. Software can be updated for different reasons and these motives can be grouped into corrective, adaptive, perfective and preventive \cite{lientz1980software}. %explain these factors a bit more closely in the related work? 

%Motivation
Because of the negative implications that modifications to to APIs cause for the API users, it is important to understand why changes to APIs happen. If most common pitfalls are known before the development of new APIs are know, these mistakes can mitigated. 

%Gap in Literature
There are no existing studies which analyse the motives behind API modifications using accepted Software Evolution theories. Studies that explore the intent behind API modifications exist \cite{hou2011exploring}, but only APIs to programming languages have been studied before.  %%add the 2nd paper that hou2011exploring intent linked to (john mentioned this) to above cite /// also add the honey language or whatever... john said "or is this a language" and googled it


%%The gap in lit section is probably written in a bitchy way... ask john for where he found stuff about how ppl positioned their study as something that awzum and new and never before done!!

\subsection{Research Questions}
\begin{itemize}
\item RQ1: What drives API evolution? 
\item RQ1.1: To what extent can we reverse engineer API change decisions?
\item SRQ1.2: Are changes/improvements to APIs made for good reasons? (Compare why the changes were made to best practises for why you should change APIs)
\end{itemize}


\section{Related Work}

\subsection{API Design}

\subsection{Software Evolution}

\subsection{Architecture / Reverse Engineering (or tracing previously made decisions)}

\section{Case Company Description}



\section{Methodology}
Brief intro, saying we will do a case study and what data collection methods will be used. (Also citing the paper from where we steal the case study process... höst).

\subsection{Case Study Design}


\subsection{Preparation for Data Collection}

\subsection{Data Gathering}

\subsection{Data Analysis}

\subsection{Threats to Validity}



\bibliographystyle{latex8}
\bibliography{latex8}


\Section{Acknowledgement}

\noindent
\end{document}

