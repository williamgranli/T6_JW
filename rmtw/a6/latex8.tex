%
%  $Description: Author guidelines and sample document in LaTeX 2.09$ 
%
%  $Author: John Burchell & William Granli $
%  $Date: 2015/01/21 15:20:59 $
%  $Revision: 1.0
%

\documentclass[10pt,twocolumn]{article} 
\usepackage{latex8}
\usepackage{authblk}

\title{Wu-Tang Clan}

% \author[*]{John Burchell}
% \author[x]{William Granli}
% \affil[*x]{Software Engineering \& Management}
% \affil[*x]{University Of Gothenburg}
% \affil[*]{john.a.burchell@gmail.com}
% \affil[x]{william.granli@gmail.com}

\begin{document}

% \author{\textsuperscript{*}John Burchell \qquad \textsuperscript{x}William Granli \\
% 		\textsuperscript{*}john.a.burchell@gmail.com \qquad \textsuperscript{x}william.granli@gmail.com \\
% 		\textsuperscript{*x} Computer Science and Engineering  \\
% 		\textsuperscript{*x}University of Gothenburg }
\author{John Burchell \qquad William Granli \\
		john.a.burchell@gmail.com \qquad william.granli@gmail.com \\
		Computer Science and Engineering  \\
		University of Gothenburg }


\maketitle





\section{Introduction}
%API Design (and a little bit of ecosystems, and a bit of problem statement)
As the software industry and the open source movement steadily grow, the number of public APIs is increasing. APIs can improve the development speed \cite{stylos2006comparing}, contribute to higher quality software \cite{stylos2006comparing} and increase the reusability of software \cite{afonso2012evaluating}. There is a consensus in that modifications to APIs could negatively impact the users of the API \cite{google_talk} \cite{mcdonnell2013empirical} \cite{robbes2012developers} \cite{henning2007api} \cite{robbes2012developers}. The main reason being that it will require the users to update the code, which is using the API, thus causing a disruption in the applicaiton's software ecosystem \cite{messerschmitt2005software}. The most common changes to APIs occur from refactoring \cite{dig2005role} \cite{xing2006refactoring}. 
%software evolution 
In the discipline of software evolution, such updates to software are studied from an evolutionary standpoint. Software can be updated for different reasons and these motives can be grouped into corrective, adaptive, perfective and preventive changes \cite{lientz1980software} \cite{iso}. 

%Gap in Literature
Studies that explore the intent behind API modifications exist \cite{hou2011exploring}. There are, however, no studies that attempt to map the motives behind changes to business needs or existing software evolution theories, such as the four categories of maintenance \cite{lientz1980software} \cite{iso}. The area of programming language APIs is well-explored \cite{hou2011exploring} \cite{shi2011empirical}, but few studies that explore platform APIs exist \cite{robbes2012developers}. To the best of our knowledge, no studies that explore the motives behind or the effects of API evolution have previously been performed on embedded platform APIs. 

%Purpose (and a bit of problem statement)
Due to the negative implications that modifications to to APIs cause for API users, it is important to understand why changes to APIs occur. If most common pitfalls are known before the development of new APIs begin, these mistakes could be mitigated. Before the industry can make use of this information, we must also critically assess to what degree we can trace such motives. To achieve this, we will perform a case study which aims to investigate the underlying factors behind API changes. This will be achieved by empirically analysing the changes between two versions of an API. Subsequently, the designers of the API will be interviewed and asked to verify the accuracy of our findings. 

%Structure of the paper
The sections following the Introduction are structured as follows: In Section 2, we present a literature review on research related to our study. Section 3 describes the case company and how it was selected. A description of our methodology is introduced in Section 4. 

\subsection{Research Questions}
\begin{itemize}
\item RQ1: What drives API evolution? 
\item RQ1.1: To what extent can we reverse engineer API change decisions?
%\item SRQ1.2: Are changes/improvements to APIs made for good reasons? (Compare why the changes were made to best practises for why you should change APIs)
\end{itemize}


\section{Related Work}
Our work is mainly related to the research area of API design. This discipline is discussed in Section 2.1. The motives and effects of changes to APIs is directly related related to the fields of software evolution and software ecosystems. This will be discussed in Section 2.2. 

\subsection{API Design}

API design is notoriously difficult as a myriad of design and performance decisions must be taken into consideration when creating APIs \cite{bloch2008effective}\cite{afonso2012evaluating}\cite{stylos2006comparing}. In previous research generalisable usability studies have been created, with the aim of analysing the design decisions taken when creating an API \cite{stylos2006comparing}. Another empirical study, which involved working with 25 programmers, highlighted API design problems such as assigning names to API features and difficulty in naming types \cite{shi2011empirical}. The study also introduced four factors which are important to consider when creating an API: a) The API  must be understandable through good documentation, b) the API must not be overly abstract, c) the API must be reusable and d) the API must be easy to learn \cite{shi2011empirical}. Both studies also highlighted that the intent of the API must be clear to the user when using design patters, such as the concrete factory pattern \cite{shi2011empirical} \cite{stylos2006comparing}.

An additional study involved combining semiotic and cognitive methods to evaluate APIs as a means of facilitating communication, between programmers and designers. \cite{afonso2012evaluating}. However, these studies do not highlight the motives behind changes to APIs.

\subsection{API Change}
Previous studies that have attempted to identify the motives behind API changes have primarily focused on APIs that are a part of large programming languages, such as Java \cite{shi2011empirical} \cite{hou2011exploring} and Smalltalk \cite{robbes2012developers}. One study attempted to uncover the intents behind the changes made in the AWT and Swing Java libraries. This particular study found that the use of a strong architecture was vital in ensuring a successful evolution of the API \cite{hou2011exploring}.

However, when APIs are changed, problems tend to occur. One study \cite{dig2005role} investigated three frameworks and one library in which the study found that 80\% of refactoring changes to APIs negatively affected existing applications. Refactorings are not the only changes to APIs that can impact existing projects negatively. Another study was conducted regarding the changes, which the researchers refer to as ripple effects \cite{robbes2012developers}. They found that changes in APIs have an effect upon the entire ecosystem. This particular empirical study found that 14\% of non-trivial API deprecations caused errors in at least one project, with the worst case of 79 projects being affected \cite{robbes2012developers}.

Furthermore, a study of the impact of reliable APIs was undertaken in the Android ecosystem \cite{mcdonnell2013empirical}. The study aimed to discover if APIs with high rates of faults affected the success of an application. The study found that the 50 least popular applications had APIs which were 500\% more error prone, ultimately concluding that there is indeed a correlation between API stability and application success \cite{mcdonnell2013empirical}. 

\section{Case Company Description}
Mention the country, size and domain. 
In domain of surveillance cameras and they develop embedded software for the cameras. 
Focus mostly on the API since that's what's relevant to our study. 
Something about market leader in their domain - and that means that changes to their API could have large consequences. 

Also try to get some information on the users of the API. (it might be relevant). %%Not sure if we can get htat now, maybe thorugh the interviews or by asking Imed

The case was selected using convenience sampling \cite{flyvbjerg2006five}, and a maximum variation strategy was used \cite{benbasat1987case}. As such, the goal was to include as many companies as possible in this study. [Note to Richard: Do you think we can claim that our case is typical, critical, revelatory or unique? Or is our current description adequate?]

%%Add why we didn't use other sampling methods for the final handin.


\section{Methodology}

\subsection{Research strategy and investigation methods}
• Describe and explain the research strategy and investigation methods that will be used for the project. Motivate your choice? Use relevant references.
AKA EXPLAIN WHY WE DO 


Brief intro, saying we will do a case study \cite{runeson2009guidelines} and what data collection methods will be used. 

Explain why case study was chosen i.e. cuz we look at something in it's natural environment (it wouldn't work to look at example data)

Another reason that the case study approach was chosen is that there is little existing research conducted in the area, and that input from the industry is vital to the success of the research. (Especially to validate if the proposed ``pitfalls''  are correct).


Explain why we chose case study compared to for example design research.
%Thius might not have to be in the proposal

The goal of the study is twofold

*Define what type of goal and explain why it is ( exploratory, descriptive, explanatory, or improving.)

*It will be an embedded case study since both the code and documentation will be used as a unit of analysis. (We can discuss if interviews will be a unit of analysis or not).

\subsection{Case Study Design and Planning}

not sure if we will need this as a section
*Do we create a case study protocol or is it too early?







\subsection{Data Collection}
• data collection procedures (e.g. interviews? survey? and sampling?) 
The main form of data collection will be performed through code and documentation analysis (third degree aka independent). In addition to this analysis, interviews will be conducted. The main purpose of the interviews is to provide insight regarding RQ1.1, but also to support data triangulation. 

Explain how we will sample the interviewees. (Ask Imed how this will be done).

It is importnat that we explain exactly why we chose the collection methods we did. It is also important htat we clearly explain which methods relate to which RQs. 

Also explain why for example questionnaire wasn't used. 


\subsection{Data Analysis}
• method for analyzing the data (grounded theory? Thematic analysis? Statistical analysis?) strategies for validating findings

How will data be analyzed?





\bibliographystyle{latex8}
\bibliography{latex8}


\Section{Acknowledgement}
John had the main responsibility for sections 1, 2. William had the main responsibility for section 3. Section 3.6 and the interview questions were written together. We have, however, worked on all sections jointly and all sections have been reviewed by both William and John. 


\noindent
\end{document}

