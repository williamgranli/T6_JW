
\documentclass[conference]{IEEEtran}
\usepackage{blindtext}

\begin{document}
\title{Exploring the Motives and Driving Forces Behind Modifications to APIs From a Software Evolution Perspective}

\author{\IEEEauthorblockN{John Burchell}
\IEEEauthorblockA{Computer Science and Engineering\\University of Gothenburg\\Gothenburg, Sweden\\
Email: john.a.burchell@gmail.com} \and
\IEEEauthorblockN{William Granli}
\IEEEauthorblockA{Computer Science and Engineering\\University of Gothenburg\\Gothenburg, Sweden\\
Email: william.granli@gmail.com}}

\maketitle


\begin{abstract}
\blindtext
\end{abstract}


\section{Introduction} \label{introduction}
%API Design (and a little bit of ecosystems, and a bit of problem statement)
As the software industry and the open-source movement steadily grow, the number of public APIs is increasing. APIs can improve the development speed \cite{stylos2006comparing}, contribute to higher quality software \cite{stylos2006comparing} and increase the reusability of software \cite{afonso2012evaluating}. There is a consensus in that modifications to APIs could negatively impact the users of the API \cite{google_talk} \cite{mcdonnell2013empirical} \cite{robbes2012developers} \cite{henning2007api}. The main reason is that it will require the users to update the code, which is using the API, thus causing a disruption in the application's software ecosystem \cite{messerschmitt2005software}. The most common changes to APIs occur from refactoring \cite{dig2005role} \cite{xing2006refactoring}. 
%Software Evolution 
In the discipline of software evolution, such updates to software are studied from an evolutionary standpoint. Software can be updated for different reasons and these motives can be grouped into corrective, adaptive, perfective and preventive changes \cite{lientz1980software}. 

%Gap in Literature
Studies that explore the intent behind API modifications exist \cite{hou2011exploring}. There are, however, no studies that attempt to map the motives behind changes to existing software evolution theories, such as the four categories of maintenance \cite{lientz1980software}. The area of programming language APIs is well-explored \cite{hou2011exploring} \cite{shi2011empirical}, but few studies that explore platform APIs exist \cite{robbes2012developers}. To the best of our knowledge, no studies that explore the motives behind or the effects of API evolution have previously been performed on embedded platform APIs. 

%Purpose (and a bit of problem statement)
%%****NEW*****
Making changes to existing APIs can have a negative impact upon the users of those APIs. It is therefore important to understand why these changes occur. Understanding the motives behind the changes could help API designers prevent or avoid potential changes to their APIs. In order to achieve this and to provide immediate benefit to industry, an exploratory case study will be undertaken to investigate the underlying motives behind API changes of an existing API. Empirical analysis of changes between two major versions of an API will be undertaken, following which an interview with the architects will be used to verify the findings of our study. 
%%****^NEW^****

Due to the negative implications that modifications to APIs cause for its users, it is important to understand why changes to them occur. If most common pitfalls are known before the development of new APIs begin, these mistakes can be mitigated. Before the industry can make use of this information, we must also critically assess to what degree we can trace such motives. To achieve this, we will perform a case study which aims to investigate the underlying factors behind API changes. This will be achieved by empirically analysing the changes between two versions of an API. Subsequently, the designers of the API will be interviewed and asked to verify the accuracy of our findings. 


\subsection{Research Questions} \label{rqs}
\begin{description}
\item[\textbf{RQ1:}] How do companies actively drive API evolution?
\item[\textbf{RQ2:}] How do APIs evolve?
\item[\textbf{RQ3:}] To what extent can we reverse engineer API change decisions?


\end{description}

%Structure of the paper
The sections following the Introduction are structured as follows: In Section \ref{existing_literature}, we present the background and previously published work that is related to our study. Section \ref{case_company_description} describes the case company and how it was selected. A description of our methodology is introduced in Section \ref{methodology}. In Section \ref{results}, the results of our study are presented, and in Section \ref{conclusion}, we discuss the findings of our study.

\section{Existing Literature} \label{existing_literature}
This section will introduce the topics of API design and software evolution, and provide a review of studies that are directly related to our study. 


\subsection{API Design} \label{api_design}
API design is notoriously difficult, as a myriad of design and performance decisions must be taken into consideration when creating APIs \cite{stylos2006comparing} \cite{afonso2012evaluating} \cite{bloch2008effective}. Examples of such decisions are whether to use the factory pattern or constructors to create objects, or if the API should display errors during compilation or runtime \cite{stylos2006comparing}. More trivial design problems, such as, assigning names to API features and difficulty in naming types also have a large impact on the usability of the API \cite{shi2011empirical}. When facing such design decisions, the following four factors are important to consider: a) The API  must be understandable through good documentation, b) the API must not be overly abstract, c) the API must be reusable and d) the API must be easy to learn \cite{shi2011empirical}. One of the most important qualities in an API is that the intent of the API is clear to the user when, for example, using design patterns, such as the concrete factory pattern \cite{stylos2006comparing} \cite{shi2011empirical}. The sum of all design decisions taken during the development of an API will decide the usability of the API. Measuring such a factor can be done by measuring the twelve cognitive dimensions which are factors that are impacted by the interaction between the API and the user of it \cite{clarke2004measuring}. 


\subsection{Software Evolution} \label{software_evolution}
%%Categories of maintenance and types of maintenance
Software evolution is a field which studies the application of software maintenance activities and processes on software, and the resulting, evolved versions of the software. The concept of software maintenance has existed since the 1960s. Well-known studies from this era established the importance of software maintenance \cite{lientz1980software}. One study would go on to have a particularly profound effect on software maintenance by establishing a set of four categories which describe the intents behind change to software \cite{lientz1980software}. These categories were refined in a further study, in which the number of categories were increased from four to twelve \cite{chapin2001types}. The categories proposed were: Enhancive, corrective, reductive, adaptive, performance, preventive, groomative, updative, reformative, evaluative, consultive and training \cite{chapin2001types}. Each category describes a type of software maintenance activity which relate to one of three particular areas; the code, the software and the customer-experienced functionality \cite{chapin2001types}. This study also helped to further the distinction between software maintenance and software evolution. The study suggested that software maintenance is a part of software evolution and that the two are not interchangeable \cite{chapin2001types}. 

%%Lehman's laws (!!!improve later)
An early study made in the area of software evolution suggests that changes to software must obey one of eight laws of software evolution \cite{lehman1980programs}. The laws presented in the study are 1) Continuing Change, 2) Increasing Complexity, 3) Self Regulation, 4) Conservation of Organisational Stability, 5) Conversation of Familiarities, 6) Continuing Growth, 7) Declining Quality and 8) Feedback System \cite{lehman1980programs}. The study also outlines that these laws only apply to E-programs, which are programs that are ``change prone" and that ``mechanise a human or societal activity" \cite{lehman1980programs}.

\subsection{Related Work} \label{related_work}
Previous studies that have attempted analyse APIs in the context of software evolution have primarily focused on APIs that are a part of large programming languages, such as Java \cite{hou2011exploring} \cite{shi2011empirical} and Smalltalk \cite{robbes2012developers}. One study \cite{hou2011exploring} attempted to uncover the intents behind the changes made in the AWT and Swing Java libraries. They found that the use of a strong architecture was vital in ensuring a successful evolution of the API \cite{hou2011exploring}. In their study, the intents were categorised around programming related factors such as reducing coupling or conformance to naming conventions.

One study \cite{dig2005role} investigated three frameworks and one library in which the study found that 80\% of refactoring changes to APIs negatively affected existing applications. Refactorings are not the only changes to APIs that can impact existing projects negatively. Another study was conducted regarding the changes, which the researchers refer to as ripple effects \cite{robbes2012developers}. They found that changes in APIs have an effect upon the entire ecosystem. This particular empirical study found that 14\% of non-trivial API deprecations caused errors in at least one project, with the worst case of 79 projects being affected \cite{robbes2012developers}.


\section{Case Company Description} \label{case_company_description}
The case company is a medium-sized company in the domain of security cameras. The company has its headquarters in Lund, Sweden, but it operates worldwide. The company is the global market leader in the markets of network cameras and video encoders. The company develops embedded software for the security cameras, and some of their products are designed to be accessible through APIs that are also developed by the company. 

%Try to get information about the users of the API. Either through Imed, or through the interviews. 

The case was selected using convenience sampling \cite{flyvbjerg2006five}, and a maximum variation strategy was used \cite{benbasat1987case}. As such, the goal was to include as many companies as possible in this study.

%Try to motivate why other sampling methods weren't used.

%Info from the interview, add to this section
Has been released, in the form we looked at (supports that we’re doing a ‘real’ case study)
They have “partners” which develop applications using their API. These applications are then used by the end-users of the camera. 

%%
Hardware
The API platform is developed for many different types of cameras. 
Some have audio, some don’t, for example.



\section{Methodology} \label{methodology}
%Intro and summary
This study has been conducted using the case study methodology \cite{runeson2009guidelines}. The goal of the study was twofold. Firstly, the aim was to empirically \textit{explore} the motives behind changes to APIs, and secondly, our ambition was to \textit{validate} the results' level of correctness by comparing our findings with the case company's explanation of their motives. Our research can be classified as an embedded case study \cite{yin2013case}, since both the API code and API documentation have been used as units of analysis. 

%Reasons for choosing case study
The reason the selected methodology was used is that it is essential to study the phenomenon of API change in its natural context. The applicability of case studies in such scenarios is supported by existing literature \cite{benbasat1987case} \cite{runeson2009guidelines} \cite{yin2013case} \cite{robson2002real}. An alternative approach that was considered was design research, but if a prototype API was to be used instead of one which is tried and tested in an industry setting, the study would lack real-life context \cite{runeson2009guidelines}. An additional motivation for why the case study approach was used is that there is little existing research conducted in the area of motives behind API change, and that input from the industry is vital to the success of the research; especially to be able to answer \textbf{RQ2}. 

%Consider adding subsub for code+doc and interviews later
\subsection{Data Collection} \label{data_collection}
%Intro to code inspection
The data were collected primarily through inspections of the source code and the archival analysis of accompanying  documentation. The source code was comprised of two versions of the same API which the company had developed. An older version which is currently released and a newer version which is soon to be deployed. Each version of the API had its own respective documentation. The documentation was inspected and organised to find explanations of the features offered by the API. 

%Description of inspection
From the APIs we collected method names, parameters and return types. (!!!Add more here. Maybe put it in a table?) The data were collected from both versions of the APIs, with the changes between the two versions being of particular interest. The collected data were input into tables and spreadsheets, according to the API version. From the documentation we collected data regarding the motive behind the functionality in API calls. Data gathered from the documentation was also stored in tables and spreadsheets and was organised together with its corresponding method.

%Reduce human error
The data collection from code and documentation was performed independently by both researchers, to reduce the risk of human errors affecting the results. If the data collected by each researcher differed, the cause of the discrepancy was investigated and resolved. 

%Interviews !!!improve later
After the initial source code inspection had been performed, interviews were conducted. The main purpose of these was to validate our findings during the code inspection with the designers of the APIs. Furthermore, the interviews were conducted to provide additional insight into decisions and motives behind changes that might have been overlooked during the code inspection. The final reason for interviewing the designers of the APIs was to provide further insight of the users of the API. 



\subsection{Data Analysis} \label{data_analysis}
%Intro and summary
The data analysis was conducted in two major phases \cite{andersson2007spiral}. During the first phase, a hypothesis generating approach \cite{seaman1999qualitative} was used to be able to fulfil the exploratory goal of the study. It was during this phase that the analysis of the API code and documentation was carried out. The second phase included analysis of the interviews, the goal of which was to confirm the hypotheses \cite{seaman1999qualitative} that were formulated in the first phase. Both phases were performed iteratively, together with the data collection. The reason for this created the opportunity to improve the process of analysis as the study progressed, as well as to be able to adapt to possible changes in direction, due to the hypothesis generating approach of the first phase \cite{andersson2007spiral}. The analysis of the code and documentation was performed using grounded theory analysis \cite{seaman1999qualitative}, as it is recommended for hypothesis generating studies \cite{runeson2009guidelines} \cite{seaman1999qualitative}.

%Description of code analysis
More specifically, the approach undertaken was divided into three steps where the data was first coded based on which category of maintenance \cite{chapin2001types} the motive of the change corresponded to. The data were then mapped and coded to the laws presented by \cite{lehman1980programs}. The codes were then analysed from a high-level point of view, the main goal of which was to identify trends in the codes. The final step involved analysis of these trends, and it was during this stage the hypotheses were generated and formulated. The same approach taken to reduce risks of research bias and human error in the data collection was performed during results analysis. 

%Description of interview analysis
Following the analysis of the archival data, the results of the interviews were analysed. The interviews were analysed using the same approach taken for the archival data. The results of this analysis was then compared and correlated with the results of the code analysis, with the goal of confirming the hypotheses that had been formulated. This support also supported triangulation of our results. 

%Relate to RQs
The software evolution categories \cite{chapin2001types} and Lehman's laws \cite{lehman1980programs} were analysed to allow us to be able to answer \textbf{RQ1}. The software evolution categories were used to map and explore motives behind changes and Lehman's laws were used to see what drives the natural evolution of APIs. Our ability to correctly predict the motives compared to the actual motives of the designers which were made clear during the interviews, will serve as the basis for answering \textbf{RQ2}.


\subsection{Validity Threats} \label{validity_threats}
In this section we discuss possible threats to construct validity, internal validity, external validity and reliability \cite{runeson2009guidelines}.

%We must add construct validity. 

%Internal Validity
\textit{Internal validity} threats for our study have been greatly reduced by the methodologies we have employed. By performing an individual inspection of  the API source code, documentation and individual results analysis, internal validity is significantly reduced. Triangulation of the results with the use of interviewing the designers is a key step in reducing potential validity threats in our study.

%External Validity
Regarding \textit{external validity}, generalisability and the appropriateness of our results are at risk. In our study we have striven to ensure that our findings will be as useful as possible to others. Given that there are some similar studies to ours and that ours is an exploratory study, our results will be useful to others who perform similar studies in the future. Our results come from both APIs either in production or soon to be in production. As such, the results of this study will hopefully be of use to the company at which they were studied, be it now for their current API or in the future when they inevitably develop others. Likewise, other companies in similar positions could use the results of our study when creating their own APIs.

%Reliability
The largest \textit{reliability} threat facing our study was the reproducibility. To counter this threat, we have followed a well-proven method for analysing the collected data. Furthermore, as mentioned previously, we have attempted to limit bias and ambiguity in our interview questions where possible. To ensure clarity of the questions, we have field tested the interview with some of our peers, improving the questions based on their feedback.


 \newpage
\section{RQ1} \label{results}
This section will cover the results.


\subsection{New}
Intro...

\subsubsection{axptz}

%%Describe the change 
Added functionality related movement of the physical camera



%%Relate it to Categories
Enable users to write applications that mechanically (or digitally) moves the camera
Requested from partners
Request ultimately came down from project management









\subsubsection{axaudio}

%%Describe the change 
Added functionality related to audio streams



%%Relate it to Categories
The most common usage of their API is to analyze video data - but some partners required audio analysis. (At least one partner did already do it, without there being an API for it).







\subsubsection{axstorage}

%%Describe the change 
Added functionality which allows connection to external storage devices
Before they just put the files in “the normal file system”



%%Relate it to Categories
Requested from partners







\subsubsection{axserialport}

%%Describe the change 
Adds functionality which allows to connect with other hardware through serial communication
Allows the possibility to integrate other systems with the camera (such as a ticket dispenser (i guess))

%%Relate it to Categories
Requested by partners 







\subsection{Modified}
Intro...

\subsubsection{axevent}

%%Describe the change 

Used to create and handle events such as detecting movement 
Split up into more files, added functionality, certain structs have been made into their own header files



%%Relate it to Categories
Re-written because of the other changes to the API (to conform with the new standards)
They were hard to use
They were marked as deprecated so that it would not break existing applications using them. 
They were however not included in the new SDK, so that it would encourage users to use the new, rewritten code







\subsubsection{axhttp}

%%Describe the change 
Converts the data so that it can be displayed on web pages
Split up into more files, added functionality (concurrency)




%%Relate it to Categories
Re-written because of the other changes to the API (to conform with the new standards)
They were hard to use
They were marked as deprecated so that it would not break existing applications using them. 
They were however not included in the new SDK, so that it would encourage users to use the new, rewritten code







\subsubsection{axparameter}

%%Describe the change 
Used to change parameters (settings) such as framerate
Not many changes, but changed so it corresponds with the new design of the API




%%Relate it to Categories
Re-written because of the other changes to the API (to conform with the new standards)
They were hard to use
They were marked as deprecated so that it would not break existing applications using them. 
They were however not included in the new SDK, so that it would encourage users to use the new, rewritten code






\subsection{Removed}
Intro...

\subsubsection{Burst Mode}

%%Describe the change 
Burst mode was deprecated
In Version 2.0 the methods only return null




%%Relate it to Categories
The feature would have to be re-implemented because of changes to the camera (hardware)
This would be costly and because of this it was investigated whether or not it would be safe to remove the feature. The investigation showed that the impact of the removal would not be significant enough and thus it was removed.






\subsection{General Trends}	

Intro...


\subsubsection{Example Code}

%%Describe the change 

Increased commenting and explanations
Improved examples that cover more functionality



%%Relate it to Categories
The partners used the examples a lot
The old examples varied a lot in how they were implemented
The old ones weren’t good enough

From what she is saying, the intent was mainly:
to have a unified standard
which helps the partners to write ‘good’ code
to cover the new functionality
to cover specific features
to show how the features can be used together





\subsubsection{Documentation}

%%Describe the change 

The documentation has mainly been changed to correspond with the changes to the API itself
LaTeX support has been removed



%%Relate it to Categories
One of the main “requests” was to increase functionality (such as ptz) - and the documentation had to be updated accordingly.








\subsubsection{Interfaces}

%%Describe the change 
They have gone from having direct access to the headers, to having a facade to access each module

The new interfaces are all based on callbacks. 
One interface that was not touched was capture.h (Except for the deprecated burst method). She thinks that it probably should have been re-implemented to bring it in line and to be consistent they should bring it into line at some point. It wasn’t required as the implementation didn’t actually change.



%%Relate it to Categories
Improvements related to coding standard. (cosmetic)
These decisions were based on internal discussion.
Wanted an event driven system
Driven by internal struggles with using the API to create applications

The need for change was identified through discussions with API users which asked why “it isn’t possible to do it this way instead etc”





\subsubsection{Error Handling}

%%Describe the change 
They now store errors in enums in separate header files
This suggests that changed their error handling to use exceptions




%%Relate it to Categories
The error handling was changed to correspond with how glib handles errors. 
The error handling should be more informative + increase usability
The change would have happened even if glib wasn’t added (but with glib made it “more obvious”)





\subsubsection{3\textsuperscript{rd} Party}

%%Describe the change 
2.0 utilizes more 3rd party libraries
axsound, for example, uses alsa
All modules use glib




%%Relate it to Categories

Architects decided to use glib, part of an internal decision. 


\newpage
\section{RQ2 Lehman}
This section will present results related to \textbf{RQ2}. The information introduced was collected during the interview with the case company. 


\subsection{1 Continuing Change — a system must be continually adapted or it becomes progressively less satisfactory}
The case company described that including new functional content was one of the top priorities for the 2.0 release. This is supported by the characteristic of the changes identified in this study. 4 of the 13 changes were additions of new modules, and in the re-implemented modules a lot of functional content was added, too. It was also mentioned that the API users had previously reqeusted more frequent updates to the API, despite the drawbacks this could have. One of the main reasons for continuously adapting the API is also to force (!!!wording) the API users to use the API in the way that it is intended. Previously, many applications which worked around the lacking (!!!negative wording) functionality had been developed. These applications were fully functional, but were developed very differently from user to user (!!!wording). We therefore conclude that Law \#1 holds for APIs. 

\subsection{2 Increasing Complexity — as system evolves, its complexity increases unless work is done to maintain or reduce it}
A large part of the changes made to the API were made to improve the structure and the way that the API is used. Many of these changes were related to the same design choices, which enforced a certain implementation style on the whole API. One example of this is the change to the interfaces which set a standard for the whole API. During the interview it was also concluded that consistency and following certain standards is an important factor to consider when designing APIs. It was also made clear a contributing factor for changing certain parts of the API was to bring it in line with the new interfaces. The interviewee mentioned that one way of achieving this was to follow guidelines or design best practices, but that the company currently did not make use of these. Even though the importance of maintaining the API to reduce complexity was stated, arguments which support that it is of lesser importance than releasing functional content was also presented. One example of where this is evident in the evolution of the API, is in the module that handles the video capturing and video stream. It was decided that this module would not be re-written to conform with the new interface standard, as there was not much functional content that would change in this module, anyways (!!!reword). Thus, Law \#2 holds for APIs, but there is evidence that suggests that decreasing the complexity is not of the highest priority. 

\subsection{3 Self Regulation — E-type system evolution processes are self-regulating with the distribution of product and process measures close to normal}
When version 1.4 was introduced, no new functionality or re-structuring was introduced. Version 1.4 only added support for additional build platforms. When comparing the changes introduced in 2.0 to the ones previously introduced, it is clear that the growth of the API has increased gradually. During the interview it was also mentioned that the future plans for the API did not include any major changes, and that only a few minor changes were planned in the near future. The future updates to the API will also be made incrementally, and there will not be a big bang update similar to the one of 2.0. This supports that Law \#3 can be applied to APIs, assuming that the current stage of the life-cycle of the API can be considered as the center of the normal distribution curve of the APIs growth. 


\subsection{4 Conservation of Organisational Stability - global work rate is invariant over the product's lifetime}
Before 1.4
No major changes have been made. 
The only change was that you could build it for more platforms

Version 1.0 was created, with incremental increases due to making other platforms available. There was only one major update which was to 2.0. In total there has only been two major updates (1.0 and 2.0).
Impression was that there was not even many official incremental builds internally. She did express earlier on in the interview that features were being created internally that ended up becoming part of the 2.0 API.


\subsection{5 Conservation of Familiarity — developers, sales personnel and users, for example, must maintain mastery of content and behaviour to achieve satisfactory evolution}
There are no current restrictions on ACAP so it’s essentially a Linux system, meaning that they can do whatever they want with it. 

Primarily the SDK is what was communicated to the partners / users of the API.

Little to no influence over the APIs. There are things that could be better than they are, for example they have to query the camera to see what it supports. She feels that they need to improve on this aspect. It’s hard to know this / do this when you’re not using the API. (This seems to suggest that they don’t use it as often internally, or that there is something missing in the documentation?)

\subsection{6 Continuing Growth — the functional content must be continually increased to maintain user satisfaction}


She reiterated that they want to add new features and modules. They want to encourage applications to be written by their partners, so if they don’t add new modules they will not have new applications created for them. When partners write applications for their cameras, the value of the cameras are enhanced so they would like to encourage partners creating new applications for their cameras.

It’s better to provide good solutions so that they don’t need to create their own custom solutions and enable good applications to be made. (Assuming that she's talking about add more functionality so that people won't have to do stuff in their own weird ways)

She was surprised that the view was that the API is more stable than they thought it was. i.e. the partners would like more changes. Firmware upgrades always worked with the API, there were very few issues.

\subsection{7 Declining Quality — the quality will appear to decline unless it is rigorously maintained and adapted}
Updates to documentation
Updates to interfaces and error handling

It's not 100\% true for APIs, because updating it too often will cause other problems --> ``rigorously''

\subsection{8 Feedback System — multi-level, multi-loop, multi-agent feedback systems}
She would like to get more feedback from the API users and the end-users - to better include this in the feedback loop and in turn use it to form change decisions

No “scientific” theories have been used when creating the APIs - most decisions have been based on experience (expertise) and previous experience

There hasn’t been anything created by the users which has then been implemented into the API. However, there are partners who have created features which were then later included. i.e. Audio (This is not to say that they used the API that the partners had developed!)



She is unaware of any metrics that they have. She would have to ask the ADP engineers for anything like this. She doesn’t personally know if there have been any issues or problems with the current API. She would like to have more feedback and input when things are working correctly, normally they only have feedback when things don’t work. Other than the feedback she mentioned previously, i.e. partners requesting features.



They had some kind of validation testing whereby they had someone internally attempt to create an application using all parts of the API. It was during the development process and was successful but it also provided them with some key feedback. She suggested that this kind of testing should be increased and that they do need more of it. (This was only one application, as it differs a lot)

Communication with users currently only really gives them bug reports, FAQs (Educational Issues) or queries about how to do something could indicate that there are missing features in the API. This is something that the product management deal with, rather than her. She focuses more on the how rather than the what. What it comes to decide what interface to do, PM decide and she then comes along and decides how it would look / be done. 

\section{RQ3 Reverse Engineer}
Table of hers and ours

And one section discussing why her answers correspond to the categories we chose.





\section{Conclusion RQ2} \label{conclusion}





%%John add conclusion ideas here


Considering the Ecosystem and considering it as a whole when designing the API. She does not think that they did take it into consideration at all. They had not thought about them when designing the APIs. She does think about who is going to use it and how but nothing in regards to how it affects an ecosystem as a whole. Things like allowing embedded webpages for the ACAP allows users to get more information from their system meaning they don’t need to ask for as much help and assistance. They usually integrate this with the VMS. Overall there was not much thought about ecosystem when the API was designed. She’s unsure if it’s one ecosystem or two. The network APIs and their API. Perhaps it is one ecosystem. (this is something Imed is looking at).
It seems that a focus on ecosystems might be starting to come through but that they already thought about them without calling them ecosystems. For example, her mentioning that they already did think about the users of the API and how they would be using it gives a slight clue that they do already think about the wider audience of their API.



%%William add conclusion ideas here
It's interesting that the partners have mentioned that they want more frequent updates... maybe API developers should be less afraid of updating them? 


\subsection{Law 1}

Problem: Leads to potentially unsafe implementations to get around shortcomings of the API

\section*{Acknowledgement} \label{acknowledgement}


\bibliographystyle{bibstyle}
\bibliography{bib}

\end{document}

